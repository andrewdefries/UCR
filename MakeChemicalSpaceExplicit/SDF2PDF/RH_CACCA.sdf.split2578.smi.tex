\chemfig{
                    O% 4
              =[:60]% 3
                       (
                       -% 5
                  -[:54]O% 6
                           (
    -[:288,0.85,,,draw=none]\mcfcringle{1.03}% (o)
                           )
                 -[:342]% 7
                           (
                     -[:270]% 11
                     -[:198]% 12
                     -[:126]% -> 5
                           )
              -[:36,,,2]^{\mcfplus}N% 8
                           (
                  -[:336,,2]\mcfright{O}{^{\mcfminus}}% 10
                           )
               =[:96,,2]O% 9
                       )
             -[:120]N% 2
             =[:180]% 1
             -[:234]S% 16
             -[:162]% 15
              -[:90]% 14
                       (
                  -[:18]N% 13
                           (
                      -[:72]% 23
                      -[:12]% 24
                      ~[:12]% 25
                           )
                 -[:306]% -> 1
                       )
             -[:150]% 20
                       (
                  -[:90]O% 21
                 -[:150]% 22
                       )
             -[:210]% 19
             -[:270]% 18
             -[:330]% 17
                       (
                  -[:30]% -> 15
    -[:150,,,,draw=none]\mcfcringle{1.3}% (o)
                       )
}
