\chemfig{
                    O% 2
             =[:270]% 1
                       (
                 -[:210]\mcfbelow{N}{H}% 10
                 -[:150]% 11
                 -[:210]% 12
                 ~[:210]% 13
                       )
             -[:330]% 3
              -[:30]% 4
                       (
    -[:270,,,,draw=none]\mcfcringle{1.3}% (o)
                       )
             -[:330]% 5
                       (
                  -[:30]F% 6
                       )
             -[:270]% 7
             -[:210]% 8
             -[:150]% 9
                       (
                  -[:90]% -> 3
                       )
}
