\chapter{Introduction}

\section{Abstract}

Biological organisms are complex, and our predecessors have forged the path to characterize how disparate components sum to a living thing. We as a community bask in this knowledge, and may share, explore and expand scientific investigation. Nonetheless, we have achieved our current understanding by reducing the complexity of living systems to strings of DNA and translation products. At present select students of Chemistry and Biology are staring into the breech between these fields and are boldly attempting to join efforts to link genes with function in biological systems \cite{saghatelian2005assignment}. These efforts are often explored as the interactions between a protein and small molecules, and expanded if possible to explain the organism at large.

\section{The technique and art of chemical genetics}

\begin{figure}
\centering
\includegraphics[scale=0.50]{TechniqueAndArt}
\caption{High throughput experiments are conducted using microwell plates and an automated robotic platform. A. Microwell plates of 96-well format contain compound libraries and are used to conduct assays. B. AP-96 multichannel fluid dispenser transfers compounds to the destination (shown as Pod1).}
\label{fig:TechniqueAndArt}
\end{figure}

In the field of chemical genetics the investigator utilizes a set of bioactive compounds to influence a biological system by inhibiting protein function on the cellular level \cite{chan2000chemical}. This approach offers the convenience to tune dosage by titrating the conecentration of the administered probe \cite{grozinger2001identification,zhao2003sir1}, and is analogous to creating a knock out mutation or knock down or leaky mutation of an allele to study protein function \cite{dai2005genetic}. A skillful investigator will use known and unknown probes to observe the phenotype spectrum or range of perturbation to gain insight into the molecular chain of events collectively referred to as mechanism of action (MOA) \cite{hayashi2003yokonolide, asami2003influence}. High throughput assays were conducted with the use of microwell plates and robotics, and enabled experimental access to thousands of experiments to test novel compounds from large chemical libraries (Figure ~\ref{fig:TechniqueAndArt}). We have used this platform to discover several novel terminal acetylene and fluorescent and tagged diazirine small molecules. In chemical genetics one may add probe at any time to the experimental system, which is form of temporal control \cite{raikhel2005adding}. I will demonstrate in Chapter 4, Phenotypes of Bioactives, how one may administer the probe as a volatile without disturbing the experimental system.


\begin{figure}
\includegraphics[scale=0.55]{ArabidopsisLinkToEukarya}
\caption{The model plant {\it Arabidopsis} is an experimental link to distant and related Eukarya. A. {\it Arabidopsis} is shown next to {\it Oryza sativa} branching from angiosperm plants long after the development of multicellularity. B. {\it Homo sapiens} has closely related organisms in animalia such as {\it Gallus gallus}, {\it Mus musculus}, and {\it Danio rerio}. C. Fungi has diverged early with {\it Saccharomyces cerevisciae} shown. Phylogenetic tree shown is unrooted and represent phylogenetic distances.}
\label{fig:ArabidopsisLinktoEukarya}
\end{figure}

\clearpage

\section{The model plant {\it Arabidopsis thaliana}}

{\it Arabidopsis thaliana}, a member of the {\it Brassicacea} is a model organism with a small sequences nuclear genome of 150 MB (Figure ~\ref{fig:ArabidopsisLinktoEukarya}A)\cite{mccourt2010arabidopsis,town2011comparative}.  A number of qualities deem {\it Arabidopsis} amenable to genetic mechanism of action studies: small size and short life cycle for ease of propagation, high progeny number to follow allele assortment, small genome to perform genetic studies approaching saturation rates, amenability to transformation \cite{FloralDip2008}, mutant resources \cite{knee2011germplasm}, ease of controlled pollination, self-pollinator, widespread geographic natural variation \cite{koornneef2011natural}. {\it Arabidopsis} is a member of the large plant family {\it Brassicacea} \cite{lysak2011phylogeny}, closely related to crops such as {\it Brassica oleracea} \cite{quiros2011genetics}, {\it Brassica napus}, and {\it Brassica rapa} \cite{ramchiary2011genetics}. The {\it Brassica} or mustard family are angiosperms that provide an agricultural staple.

%Get zoom of brassica

\section{{\it Arabidopsis} for mechanism of action studies}

\begin{figure}
\includegraphics[scale=0.50]{ArabidopsisAndMan}
\caption{The life cycle of the model organisms enables us to translate biological understanding to more complex organisms. A-E. Fertilization, embryonic development juvenille development, and senescence are common developmental milestones between {\it Arabidopsis thaliana} and {\it Homo sapiens}. BBCH codes for the developmental stage (explained in Chapter 4: Phenotypes of Bioactives) are shown for {\it Arabidopsis} from 00-99.}
\label{fig:ArabidopsisAndMan}
\end{figure}

Genetic redundancy that accumulated during the evolution of {\it Arabidopsis} was lost during the appearance of {it A. thaliana} \cite{parkin2011chasing}. {\it Arabidopsis} has a small genome with half the ploidy of the ancestor {\it Arabidopsis lyrata} \cite{savolainen2011arabidopsis}, {\it Capsella} \cite{theissen2011genetics}, and other {\it Brassicacea}. Functional redundancy \cite{pickett1995seeing} has been a fundamental problem \cite{bouche2001arabidopsis} for the uncovering of conserved gene networks in a number of model organisms \cite{cutler2005dude}. Therefore, {\it Arabidopsis thaliana} has been chosen, in this thesis, as the model organism for experimentation. 

\section{{\it Arabidopsis}, an experimental link to distant and related Eukarya}

The life cycle of {\it Arabidopsis} can be used to study developmental processes that have parallels in {\it Homo sapiens} (Figure ~\ref{fig:ArabidopsisAndMan}A-E) \cite{boyes2001growth}. For instance, both organisms undergo meiosis, sexual reproduction, fertilization, embryonic, juvenile, adult development, and senescence. Beyond these developmental similarities a comparison of the functional categories of {\it Arabidopsis} proteins compared to other sequenced genomes revealed 48-60{\%} {\it Arabidopsis} proteins involved in proteins synthesis with counterparts in other species \cite{primrose2009principles}. This is a reflection of highly conserved gene functions between plants and related organisms. 

%consider listing
%214 is primrose

\section{Chemical genetics in the model plant {\it Arabidopsis thaliana}}

In the elucidation of genetic networks essential genes can cause embryonic lethal phenotypes.  The technique of chemical genetics offers a means to ablate gene function after the passage of developmental milestones to expose a phenotype spectrum for examination \cite{briggs2006unequal} evidenced by recent findings \cite{park_2009}. A number of investigators have successfully employed simple target identification strategies for small molecules using classical genetic techniques. Thus, {\it Arabidopsis} is poised for use in chemical genetics research.

%consider citing table (Hypostatin, Brz, Sirtinol, Pyrabactin, Tir1)

\section{Vision of tagged chemical libraries}

During the infancy of chemical genetics, what was considered by Mitchison and Schreiber as pharmacological genetics \cite{mitchison1994towards}, it was a new milieu to use lead compounds in the same manner as genetic mutations to identify previously unknown targets. This was envisioned to be accomplished efficiently through covalent bond formation between ligand and receptor, an atypical situation. The concept was to produce combinatorial amine-like libraries that possessed functional groups to enhance downstream investigation of bioactives via affinity enrichment or covalent capture or both \cite{mitchison1994towards}. These methods were born out of the problems that arise in forward chemical genetics from novel small molecule probes that produce interesting phenotypes but require a high concentration to achieve 50{\%} inhibition of the observed phenotype (IC50) in the range of 0.5 nM to 25 $\mu$M. 

\section{Activity based proteomic profiling, a method to create more tractable small molecule probes}

Chemical genetics has a numerous problems that can complicate downstream investigation of bioactive leads \cite{burdine2004target}. Activity Based Proteomic Profiling (ABPP) has enhanced our odds of successful biochemical isolation of targets through the exploitation of irreversible inhibitors \cite{saghatelian2005assignment}. In ABPP the desired probe is bioactive either {\it in vivo} or {\it in vitro} and possesses reactive functional group(s) for a specific family of enzymes \cite{adibekian2011click}. These probes are also thoughtfully designed to be tri-functional with a bioactivity group, a fluorescent or other detection group, and a group for biotin-avidin affinity chromatography techniques \cite{walsh2006chemical}.

%consider listing some probes and the type of groups

\section{Comparison of probes in ABPP and forward chemical genetics}

\begin{figure}
\includegraphics[scale=0.50]{LinkerStrategy}
\caption{Chemical diversity in nature has inspired synthetic chemistry and combinatorial chemistry. A. A rich chemical diversity is possible from the oligomers DNA, peptides and proteins. B. The generic di-peptide shown here was formed from reaction with the amino or N-terminal and carboxy or C-terminal of two amino acids to produce the characteristic peptide backbone. C. Peptoids are a synthetic attempt to mimic a diverse peptide backbone and cap the reactive N and C-terminal ends. D. Our amine azide linker strategy is amenable to click chemistry. E. Introduction of diverse substituents to the backbone (Shown as R2 and R3) forms a peptide bond isostere, the triazole ring.}
\label{fig:CombiChemfromOligomers}
\end{figure}

The type of probes used in ABPP are not of interest to the forward chemical genetics investigator unless they can be synthesized in large diverse libraries \cite{dobson2004chemical}, screened {\it en masse} \cite{schreiber2003small,schreiber1998chemical} and stored. Small molecule probes containing reactive groups are not the end goal of forward genetics. This is why we seek, in this thesis, a probe that can be activated by ultraviolet (UV) light to produce a carbene {\it in situ}. It is our hope that through diazirine photolysis a carbene functionalized probe can be utilized for covalent labeling and facile downstream target investigation \cite{blencowe2005development,bond2009photocrosslinking}. We have devised a strategy to prepare and screen a combinatorial library to find this probe to promote the vision of pharmacological genetics. The design of the amine building block and use for generation of combinatorial libraries will be detailed (Chapter 2: Organic synthesis), and biological use will also be detailed (Chapter 3: Chemical genetic screen).  

\section{Diverse chemical libraries from combinatorial reactions}

Combinatorial libraries for chemical genetics have been inspired by the simplicity and diversity of biomolecules derived from natural amino acids (Figure ~\ref{fig:CombiChemfromOligomers}A) \cite{pirrung2004molecular}. Peptide combinatorial libraries innovated a number of high throughput solid phase techniques and paved the way for the synthesis of large libraries of peptide-like small molecules \cite{simon1992peptoids,dorner1996synthesis}. Peptoids, unnatural oligomers from N-substituted glycine, have a similar side chain spacing as peptides but lack stereochemistry (Figure ~\ref{fig:CombiChemfromOligomers}C)\cite{simon1992peptoids}. Our rationale was motivated by the fact that the formation of a 1,4 substituted 1,2,3-triazole (Figure ~\ref{fig:CombiChemfromOligomers}E) was considered a peptide isostere [6] and enabled a wide range of reactants using a polytriazole catalyst \cite{chan2004polytriazoles} and reduced copper \cite{himo2005copper}. Therefore, we sought to design a small azide with a secondary amine for fluorophore conjugation, and an amine with a capped acetamide for combinatorial library generation using a reaction in water that is high yielding and regiospecific known as click chemistry (Figure ~\ref{fig:CombiChemfromOligomers}D). Later bioactive leads may be functionalized into carbene generating Photo Affinity Ligands (PALs) using a boc-protected amine azide building block (synthesis discussed in detail in Chapter 2: Organic Synthesis and use discussed in detail in Chapter 6: Target identification efforts using covalent capture).

\begin{figure}
\includegraphics[scale=0.50]{ChangTriazine}
\caption{The purine scaffold was used to develop several combinatorial libraries. A. A purine ring scaffold shown with a number of diversity generation sites. B. Diminutol is a bioactive microtubule dynamics regulator. C. Myoseverin is also a microtubule binding compound. D. A triazine scaffold was used for the generation of related combinatorial libraries. E. Encephalazine is a bioactive in {\it Danio rerio} embryos. F. Atrazine is a popular herbicide.}
\label{fig:TriazineTaggedLibraries}
\end{figure}

\section{Triazine tagged combinatorial libraries}

The specific design of our amine-azide takes into account previous studies in forward chemical genetics with libraries from a purine scaffold (Figure~\ref{fig:TriazineTaggedLibraries}A) \cite{wignall2004identification}. Investigators made successive advancements with the purine scaffold to identify more novel bioactives (Figure~\ref{fig:TriazineTaggedLibraries}A-C), and later learned the switch to a trizine core would reveal novel bioactives. These triazine compounds possessed groups for immobilization on an agarose support and were used to identify protein targets in the model organism {\it Danio rerio}, {\it in vitro} human cell lines \cite{moon2002novel,williams2004identification,ni2005triazine}, and in {\it Xenopus laevis} extracts \cite{wignall2004identification} (Figure ~\ref{fig:ArabidopsisLinktoEukarya}B, clade tip not shown).

The genius of the approach was to synthesize and screen libraries on the same scaffold with and without linkers. Therefore, investigators could clearly identify leads amenable to modification. The triazine scaffold was not sufficiently diverse or amenable for use in plant chemical genetics since it resembles the popular herbicide atrazine. Our chemistry platform capitalizes on commercially available precursors for amine block synthesis and widely available drug-like terminal acetylenes for library synthesis.


%JBS: Outline the objectives of the dissertation by data chapter. Make a systematic list of what was done in each, explain briefly.

\section{Conclusion}

Our work was accomplished through the union of automation tools, cheminformatics, image analysis, spectroscopy, spectrometry, organic synthesis, pharmacology and genetics. These tools, examples, and bundled R code can be used to perform chemical genetics in {\it Arabidopsis} and offer inspiration for work in other model organisms.


