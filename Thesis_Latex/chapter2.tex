\chapter{Organic Synthesis}

\section{Abstract}

We envisioned a combinatorial chemistry strategy that enabled efficient attachment of diverse pharmacophores such as drug-like terminal acetylenes with fluorescent building block azides using an amine azide linker. The acetamide half of the amine azide linker was synthesized from glycine, and subjected to reductive amination to produce building block A. The attachment of fluorophores was achieved using standard amine coupling techniques with several flourophores. A second building block was synthesized that enables the derivitization of the hit molecule to a diazirine functionalized fluorescent probe. Furthermore, we synthesized a unique tris-[(1-benzyl-1H-1,2,3-triazol-4-yl)methyl]amine (TBTA) catalyst that enabled direct biological screening of combinatorial libraries synthesized in high throughput via the click reaction with the clickable library and fluorescent amine azide building blocks. 

\section{Introduction}

\begin{figure}
\centering
\includegraphics[scale=0.35]{ClickChemistryMechanism_cleaned}
\caption{Copper catalyzed ligand mediated click chemistry.}
\label{fig:ClickChemistry}
\end{figure}

The azide-alkyne copper catalyzed ligand-mediated \cite{chan2004polytriazoles} Huisgen 1,3-dipolar cycloaddition known as click chemistry \cite{kolb2001click} creates products in high yield \cite{himo2005copper}, with high regioselectivity, can be performed in water \cite{kolb2001click}, and is a general reaction with azides and alkynes (Figure~\ref{fig:ClickChemistry}). In this variant of click chemistry a ligand and reduced copper coordinate with the acetylene and azide to catalyze the formation of the 1,2,3-triazole (Figure~\ref{fig:ClickChemistry}). Therefore, we chose click chemistry as our platform for high throughput synthesis. Our approach is feasible since reactions with acetylenes and azides were characterized as bioorthogonal \cite{agard2006comparative}, amenable to high throughput, and commercial availability of approximately 80,000 drug-like compounds with terminal acetylenes \cite{irwin2005zinc}.

\clearpage

\section{Materials and Methods}


\subsection{2-azidoethan-1-amine}

\begin{figure}
\centering
\includegraphics[scale=0.25]{AEA}
\caption{2-azidoethan-1-amine.}
\label{fig:AEA_Synth}
\end{figure}

2-azidoethan-1-amine (Figure~\ref{fig:AEA_Synth}) was synthesized by dissolving sodium azide (16.8 g, 258 mmol) and 2-chloroethan-1-amine hydrochloride (10 g, 86 mmol) into water (86 ml) and heating for 15 hours at 80${}^\circ$ C. The reaction was cooled to room temperature and KOH was added until the pH was ~8.5. Thereafter, the product was extracted into ice cold diethyl ether (200 ml) from the aqueous solution, the organic was dried over anhydrous Na2SO4. Solvent was removed {\it in vacuo} to obtain a pure light yellow oil (3.7 g, yield 50{\%}). \\

\noindent 
1HNMR (300 MHz, CDCl3): delta = 1.35-1.50 (br s, 2H), 2.89 (t, {\it J} = 6 Hz, 2H), 3.37 (t, {it J} = 6 Hz, 2H). \\
IR(neat):2090 cm-1 (azide stretching).

\clearpage


\subsection{N-acetyl-N-(2-oxopropyl)acetamide (Dakin-West).}

\begin{figure}
\centering
\includegraphics[scale=0.25]{DakinWest}
\caption{N-acetyl-N--(2-oxopropyl)acetamide.}
\label{fig:DakinWestSynth}
\end{figure}

N-acetyl-N--(2-oxopropyl)acetamide (Figure~\ref{fig:DakinWestSynth}) was synthesized from glycine (75 g, 1.0 mol), acetic anhydride (1.1 L, 12 mol) and pyridine (485 ml, 6.0 mol) was refluxed overnight under positive pressure with a gentle stream of N2 bubbling from a needle at the bottom of the vessel or with a N2 balloon. Solvents were removed {\it in vacuo} to obtain a viscous black crude. Deacetylation of N-acetyl-N-(2-oxopropyl) acetamide to N-(2-oxopropyl)acetamide was accomplished through gentle reflux in excess methanol overnight. Volatiles were removed {\it in vacuo} to obtain a brown-red oil with a sweet smell. The residue was further purified via Kugelrorh distillation at 114${}^\circ$ C under reduced pressure (0.7 Torr). The resulting white oil was purified by silica chromatography on a gradient elution of DCM:isopropanol from 98:2 to 90:10 in five steps (1 g, yeld ~1{\%}). \\

\noindent
1HNMR (400 MHz, CDCl3): delta = 2.05 (s, 3H), 2.22 (s, 3H), 4.17 (d, {\it J} = 4.2 Hz, 2H), 6.19 (br s, 1H).

\clearpage

\subsection{N-N-2[(2-azidoethyl)amino]propylacetamide (block A)}

\begin{figure}
\centering
\includegraphics[scale=0.25]{BlockA}
\caption{N-N-2[(2-azidoethyl)amino]propylacetamide.}
\label{fig:BlockASynth}
\end{figure}

N-N-2[(2-azidoethyl)amino]propylacetamide (block A) (Figure~\ref{fig:BlockASynth}) was synthesized from 2-azidoethan-1-amine 1.6 g, 20 mmol), N-(2-oxopropyl)acetamide (2.3 g, 20 mmol) and LiCl (1.2 g, 30 mmol). The reactants were added to methanol (25 ml) and stirred for 1h before the addition of sodium cyanoborohyrdide (1.3 g, 20 mmol), and the reaction was left to proceed overnight. Solvent was removed {\it in vacuo} and the crude was washed with saturated aqueous NaHCO3, extracted into EtOAc, filtered over a cotton plug, dried with anhydrous Na2SO4, and evaporated {\it in vacuo}. The resultant crude product was dissolved in chloroform and filtered over a cotton plug to remove insoluble contaminants to obtain the crude product (1.4 g, yield 39{\%}). \\

\noindent
1HNMR (300 MHz, CDCl3): delta = 1.08 (d, {\it J} = 6.3 Hz, 3H), 1.99 (s, 3H), 2.69-2.84 (m, 2H), 2.91-3.05 (m, 2H), 3.33-3.43 (m, 3H), 6.08 (br s, 1H). \\

\noindent
13CNMR (75 MHz, CDCl3): delta = 18.4. 23.0, 44.5, 45.9, 51.5, 52.4, 171.0.\\

\clearpage

\subsection{Synthesis of tert-butyl N-\{2-[(2-azidoethyl)amino]propyl)carbamate block B}

\begin{figure}
\centering
\includegraphics[scale=0.25]{BocBlockB}
\caption{tert-butyl N-\{2-[(2-azidoethyl amino]propryl\}carbamate (block B).}
\label{fig:BocBlockBSynth}
\end{figure}

Suvadeep Nath synthesized tert-butyl N-(2-oxopropyl)carbamate from an amino alcohol precursor and this was subjected by him to reductive amination similar to block A to produce tert-butyl N-\{2-[(2-azidoethyl amino]propryl\}carbamate (Figure~\ref{fig:BocBlockBSynth}), otherwise referred to as boc-protected block B (Figure 2.7). NMR analysis was performed by Andrew Defries. \\

\noindent
1HNMR (300 MHz, CDCl3): delta = 1.07 (d, {\it J} = 6.3 Hz, 3H), 1.46 (s, 9H), 2.71-3.03 (m, 4H), 3.12-3.23 (m, 1H), 3.38-3.43 (m, 2H), 4.96 (br s, 1H). \\

\noindent
13CNMR (75 MHz, CDCl3): delta = 18.6, 28.5, 45.6, 46.0, 51.9, 52.8, 79.2, 156.4. \\

\clearpage

\subsection{Dansyl block A}

\begin{figure}
\centering
\includegraphics[scale=0.25]{DansylBlockA}
\caption{Dansyl block A.}
\label{fig:DansylBlockASynth}
\end{figure}

Dansyl block A was synthesized ((Figure~\ref{fig:DansylBlockASynth}) by adding dansyl-Cl (5.0 g, 18 mmol), DIPEA (7.2 g, 56 mmol), a catalytic amount of DMAP, and block A (3.3 g, 18 mmol) in dry DCM under an N2 atmosphere. Volatiles were removed {\it in vacuo} and the residue was purified by SiO2 chromatography using a DCM:isopropanol gradient elution. Alternatively, the crude may be dissolved in 50:50 ACN:H2O and purified on a CombiHT-C8 preparative Reverse-Phase column and eluted using an isocratic method with 50:50 (solvent A 95{\%} water, 5{\%} ACN, 0.05 {\%} formic acid and solvent B 95{\%} ACN, 5{\%} water, 0.05{\%} formic acid) (300mg, yield 1{\%}). \\

\noindent
1HNMR (500 MHz, CDCl3) delta = 0.96 (d, {\it J} = 6.5 Hz, 3H), 1.64 (s, 3H), 2.89 (s, 6H), 3.01 (m, 1H), 3.22 (m, 1H), 3.38 (m, 1H), 3.54 (m, 2H), 3.69 (m, 1H), 3.89 (m, 1H), 5.94 (br s, 1H), 7.21 (d, {\it J} = 7 Hz, 1H), 7.56 (dd, {\it J} = 1 Hz, {\it J} = 7.5 Hz, 1H), 7.62 (dd, {\it J} = 1 Hz, {\it J} = 7.5 Hz, 1H), 8.27 (d, {\it J} = 9 Hz, 1H), 8.30 (dd, {\it J} = 1 Hz, {\it J} = 7.5 Hz, 1H), 8.59 (d, {\it J} = 8.5 Hz, 1H). \\

\noindent
m/z of 419.1866 was obtained for dansyl block A corresponding to (M+H) for the expected molecular formula C19H26N6O3S. \\

\clearpage

\subsection{Dansyl boc block B}

\begin{figure}
\centering
\includegraphics[scale=0.25]{DansylBocBlockB}
\caption{Dansyl boc block B.}
\label{fig:DansylBocBlockBSynth}
\end{figure}

Dansyl block B (Figure~\ref{fig:DansylBocBlockBSynth}) was synthesized by adding dansyl-Cl (5.0 g, 18 mmol), DIPEA (7.2 g, 56 mmol), a catalytic amount of DMAP, and block B (4.4 g, 18 mmol) in dry DCM under N2 atmosphere (Figure S14). Volatiles were removed {\it in vacuo} and the residue was purified similarly to block A dansyl on an isocratic method with 30:70 (solvent A: solvent B) as eluent. \\

\noindent
1HNMR (400 MHz, CDCl3): delta = 0.92 (d, {\it J} = 6.8 Hz, 3H), 1.42 (s, 9H), 2.90 (s, 6H), 3.0-3.06 (m, 2H), 3.10-3.30 (m, 2H), 3.44-3.52 (m, 2H), 3.90-4.1 (m, 1H), 4.99 (br s, 1H), 7.20 (d, {\it J} = 8 Hz, 1H), 7.54 (t, {\it J} = 7.6 Hz, 1H), 7.60 (t, {it J} = 8.4 Hz, 1H), 8.27-8.31 (m, 2H), 8.57 (d, {\it J} = 8.4 Hz, 1H). \\

\noindent
m/z of 477.2268 was obtained corresponding to (M+H)+ for the expected molecular formula C22H32N6O4S. \\

\clearpage

\subsection{Dansyl free amine block B}

\begin{figure}
\centering
\includegraphics[scale=0.25]{DansylFreeAmineBlockB}
\caption{Dansyl free amine block B.}
\label{fig:DansylFreeAmineBlockB}
\end{figure}

The resulting boc precursor which was de-protected (Figure~\ref{fig:DansylFreeAmineBlockB}) using 10{\%} TFA in DCM at room temp for 1h. Volatiles were removed {\it in vacuo} and the free amine was isolated by preparative RP-HPLC. \\

\noindent
1HNMR (400 MHz, D6-DMSO): delta = 0.84 (d {\it J} = 6.4 Hz, 3H), 2.85 (s, 6H), 2.92-2.95 (m, 1H), 3.29-3.34 (m, 1H), 3.47-3.61 (m, 4H), 4.08-4.14 (m, 1H), 7.32 (d, {\it J} = 8 Hz, 1H), 7.63-7.70 (m, 2H), 8.22 (d, {\it J} = 6.4 Hz, 1H), 8.30 (d, {\it J} = 8.4 Hz, 1H), 8.54 (d, {\it J} = 8 Hz, 1H). \\

\noindent
13CNMR (100 MHz, D6-DMSO): delta = 16.30, 22.42, 38.16, 42.90, 45.10, 45.75, 49.94, 50.40, 52.69, 114.07, 116.97, 119.05, 123.75, 128.25, 129.18, 130.00, 135.05, 151.08, 158.30, 169.40. \\

\noindent
m/z of 377.1757 was obtained corresponding to (M+H)+ for the molecular formula C17H24N6O2S. \

\clearpage

\subsection{Synthesis of diazirine dansyl block B (DDA)}

\begin{figure}
\centering
\includegraphics[scale=0.25]{DansylDiazirine}
\caption{Dansyl diazirine.}
\label{fig:DansylDiazirineSynth}
\end{figure}

Equimolar amounts of (0.22 mmol) SDA-Diazirine (Pierce) and dansyl block B free amine (83 mg, 0.22 mmol) were dissolved in a minimum of DMSO and diluted with DCM/MeOH containing DIPEA. The reaction was conduced overnight and the product was purified by RP-HPLC using an 80{\%} isocratic method using ACN/H2O) as eluent to give the product dansyl diazirine block B (Figure~\ref{fig:DansylDiazirineSynth})(40 mg, 50{\%} yield). \\

\noindent
1HNMR (500 MHz, D6-DMSO): delta = 0.93 (d, {\it J} = 7 Hz, 3H), 1.56 (s, 3H), 2.84 (s, 6H), 3.08 (m, 2H), 3.3-3.5 (m, 9H), 3.95 (m, 1H), 7.27 (d, {\it J} = 7.5 Hz, 1H), 7.61-7.66 (m, 2H), 8.15 (d, {\it J} = 9 Hz, 1H), 8.18 (dd, {\it J} = 1 Hz, {\it J} = 7.5 Hz, 1H), 8.50 (d, {\it J} = 8.5 Hz, 1H). \\

\noindent
13CNMR (126 MHz, D6-DMSO): delta = 16.26, 22.32, 39.99, 41.71, 42.05, 45.03, 50.38, 52.65, 115.21, 118.74, 123.65, 128.23, 129.21, 129.76, 130.16, 134.96, 151.43, 169.21. \

\noindent
m/z of 487.2234 was obtained corresponding to (M+H)+ for the expected molecular formula C22H30N8O3S. 

\clearpage

\subsection{TBTA-(CO2Me)3}

\begin{figure}
\centering
\includegraphics[scale=0.25]{TBTACOOMe}
\caption{TBTA-(CO2Me)3.}
\label{fig:TBTACOOMe}
\end{figure}

Synthesis of TBTA-(CO2Me)3 was performed by Suvadeep Nath (Figure~\ref{fig:TBTACOOMe}). NMR analysis was performed by Andrew Defries. \\

\noindent
1HNMR (400 MHz, D6-DMSO): delta = 3.32 (s, 9H), 3.65 (s, 6H), 5.70 (s, 6H), 7.37 (d, {\it J} = 8.8 Hz, 6H), 7.94 (d, {\it J} = 7.6 Hz, 6H), 8.13 (s, 3H). \\

\noindent
13CNMR (100 MHz, CDCl3): delta = 47.68, 52.8, 52.9, 125.2, 128.5, 129.9, 130.3, 142.2, 144.5, 166.5. \\

\clearpage

\subsection{TBTA-(CO2H)3}

\begin{figure}
\centering
\includegraphics[scale=0.25]{TBTACOOH}
\caption{TBTA-(CO2H)3.}
\label{fig:TBTACOOH}
\end{figure}

Synthesis of TBTA-(CO2H)3 was performed by Suvadeep Nath (Figure~\ref{fig:TBTACOOH}). NMR analysis was performed by Andrew Defries. \\

\noindent
1HNMR (300 MHz, D6-DMSO): delta = 3.31 (br s, 3H), 3.65 (s, 6H), 5.67 (s, 6H), 7.34 (d, {\it J} = 8.4 Hz, 6H), 7.91 (d, {\it J} = 7.8 Hz, 6H(, 8.12 (s, 3H). \\

\noindent
13CNMR (100 MHz, CDCl3): delta = 47.7, 53.0, 125.15, 128.4, 130.4, 131.3, 141.6, 144.4, 167.6. \\

\clearpage

\subsection{TBTA-(CO2Na)3}

\begin{figure}
\centering
\includegraphics[scale=0.25]{TBTACOONa}
\caption{TBTA-(CO2Na)3.}
\label{fig:TBTACOONa}
\end{figure}

Synthesis of TBTA-(CO2Na)3 was performed by Suvadeep Nath (Figure~\ref{fig:TBTACOONa}). NMR analysis was performed by Andrew Defries. \\

\noindent
1HNMR (400 MHz, D20): delta = 3.67, (s, 6H), 5.33 (s, 6H), 7.08 (d, {\it J} = 8.0 Hz, 6H), 7.64 (s, 3H), 7.67 (d, {\it J} = 8.4 Hz, 6H). \\

\noindent
13CNMR (100 MHz, D20): delta = 48.2, 53.4, 125.3, 127.7, 129.6, 136.6, 137.7, 144.1, 174.8. \\


\clearpage

\subsection{CLK021C05 dansyl block A}

\begin{figure}
\centering
\includegraphics[scale=0.25]{CLK021C05_DansylBlockA}
\caption{CLK021C05 dansyl block A}
\label{fig:CLK021C05_DansylBlockA}
\end{figure}

Synthesis of CLK021C05 dansyl bock A (Figure~\ref{fig:CLK021C05_DansylBlockA}) and the subsequent compounds were synthesized using standard click chemistry conditions using reduced copper and TBTA-(CO2Na)3 with the respective azide building block and terminal acetylene from the clickable collection. \\

\noindent
1HNMR (400 MHz, CD3OD): delta = 0.87 (d, {\it J} = 6.8 Hz, 3H), 1.66 (s, 3H), 2.63 (t, 8H), 2.84 (s, 6H), 2.92-3.19 (m, 6H), 5.12 (s, 2H), 7.06 (d, {\it J} = 9.2 Hz, 1H), 7.23-7.31 (m, 7H), 7.37 (d, {\it J} = 8.4 Hz, 2H), 7.55-7.62 (m, 2H), 7.85 (s, 1H), 8.23 (d, {\it J} = 8.8 Hz, 1H), 8.26 (dd, {\it J} = 7.6 Hz, 1H), 8.52 (br s, 1H), 8.59 (d, {\it J} = 8.4 Hz, 1H). \\

\noindent
m/z of 706.3146 was obtained corresponding to the (M+Na)+ for the expected molecular formula C37H45N7O4S. \\

\clearpage

\subsection{CLK024F02 dansyl block A}

\begin{figure}
\centering
\includegraphics[scale=0.25]{CLK024F02_DansylBlockA}
\caption{CLK024F02 dansyl block A.}
\label{fig:CLK024F02_DansylBlockA}
\end{figure}

\noindent
(Figure~\ref{fig:CLK024F02_DansylBlockA})

\noindent
1HNMR (400 MHz, D6-DMSO): delta = 0.83 (d, {\it J} = 6.8 Hz, 3H), 1.53 (s, 3H), 2.54 (s, 1H), 2.83 (s, 6H), 3.00 (t, {\it J} = 7.6 Hz, 2H), 3.60-3.64 (m, 2H), 3.90 (m, 1H), 4.50 (t, {\it J} = 7.2 Hz, 2H), 4.64 (s, 2H), 7.26 (d, {\it J} = 7.6 Hz, 1H), 7.37 (dd, {\it J} = 2.0 Hz, {\it J} = 8.8 Hz, 1H), 7.57-7.65 (m, 2H), 7.70 (d, {\it J} = 8.8 Hz, 1H), 7.77 (d, {\it J} = 2 Hz, 1H), 8.10 (s, 1H), 8.14 (d, {\it J} = 9.2 Hz, 1H), 8.17 (dd, {\it J} = 1.2 Hz, {\it J} = 7.6 Hz, 1H), 8.50 (d, {\it J} = 8.4 Hz, 1H). \\

\noindent
m/z of 642.1698 was obtained corresponding to the (M+H) for the expected molecular formula C29H32ClN7O4S2. \\

\clearpage

\subsection{CLK024F02 dansyl diazirine}

\begin{figure}
\centering
\includegraphics[scale=0.25]{CLK024F02_DansylDiazirine}
\caption{CLK024F02 dansyl diazirine.}
\label{fig:CLK024F02_DansylDiazirine}
\end{figure}

\noindent
(Figure~\ref{fig:CLK024F02_DansylDiazirine})

\noindent
1HNMR (400 MHz, CDCl3): delta = 0.84 (dm {\it J} = 7.2 Hz, 3H), 1.55-1.80 (m, 7H), 2.91 (s, 6H), 2.95-3.10 (m, 1H), 3.18-3.25 (m, 1H), 3.37-3.48 (m, 1H), 3.49-3.60 (m, 2H), 3.60-3.76 (m, 1H), 3.85-4.00 (m, 1H), 4.59 (s, 2H), 7.20 (d, {\it J} = 7.2 Hz, 1H), 7.24 (dd, {\it J} = 6.4 Hz, {\it J} = 2.0 Hz, 1H), 7.37 (d, {\it J} = 4.0 Hz, 1H), 7.40 (d, {\it J} = 3.6 Hz, 1H), 7. 52-7.60 (m, 3H), 7.64 (d, {\it J} = 2 Hz, 1H), 8.14 (d, {\it J} = 8.4 Hz, 1H), 8.25 (dd, {\it J} = 7.6 Hz, {\it J} = 0.8 Hz, 1H), 8.58 (d, {\it J} = 8.4 Hz, 1H). \\

\noindent
m/z of 709.2020 was obtained corresponding to a loss of N2, CH3, and 2H for the expected molecular formula C32H36ClN9O4S2. \\

\clearpage

\subsection{CLK042A09 dansyl block A}

\begin{figure}
\centering
\includegraphics[scale=0.25]{CLK042A09_DansylBlockA}
\caption{CLK042A09 dansyl block A.}
\label{fig:CLK042A09_DansylBlockA}
\end{figure}

\noindent
(Figure~\ref{fig:CLK042A09_DansylBlockA})

\noindent
1HNMR (400 MHz, D6-DMSO): delta = 0.92 (d, {\it J} = 6.8 Hz, 3H), 1.55 (s, 3H), 2.62 (d, {\it J} = 4.8 Hz, 2H), 2.82 (s, 6H), 3.01-3.05 (m, 2H), 3.17 (s, 3H), 3.33 (s, 2H), 3.56-3.61 (m, 2H), 3.92-3.97 (m, 1H), 4.57 (s, 2H), 4.64 (d, {\it J} = 5.6 Hz, 2H), 7.65 (q, {\it J} = 9.6 Hz, {\it J} = 4.4 Hz, 3H), 7.24-7.41 (m, 5H), 7.58-7.69 (m, 2H), 7.96 (dd, {\it J} = 66 Hz, {\it J} = 8.8 Hz, 4H), 8.14 (d, {\it J} = 8.8 Hz, 1H), 8.18 (dd, {\it J} = 7.2 Hz, {\it J} = 1.2 Hz, 1H), 8.26 (br s, 1H), 8.50 (d, {\it J} = 8.4 Hz, 1H), 9.20 (t, {\it J} = 11.6 Hz, 1H). \\
%66 or 6.6 Hz

\noindent
m/z of 854.2782 was obtained corresponding to (M+H)+ for the expected molecular formula C38H47N9O8S3. \\

\clearpage

\subsection{CLK042A09 dansyl diazirine}

\begin{figure}
\centering
\includegraphics[scale=0.25]{CLK042A09_DansylDiazirine}
\caption{CLK042A09 dansyl diazirine.}
\label{fig:CLK042A09_DansylDiazirine}
\end{figure}

\noindent
(Figure~\ref{fig:CLK042A09_DansylDiazirine})

\noindent
1HNMR (300 MHz, CDCl3): delta = 0.83 (d, {\it J} = 6.6 Hz, 3H), 1.57 (s, 3H), 1.60-1.90 (m, 4H), 2.84 (d, {\it J} = 5.4 Hz, 2H), 2.91 (s, 6H), 3.00-3.10 (m, 2H), 3.45-3.58 (m, 2H), 3.62-3.75 (m, 2H), 3.84-4.23 (m, 2H), 4.49 (s, 2H), 4.73 (d, {\it J} = 5.1 Hz, 3H), 5.81 (s, 1H), 6.07 (t, {\it J} = 9.3 Hz, 1H), 7.11 (s, 1H), 7.23 (d, {\it J} = 7.8 Hz, 1H), 7.32-7.41 (m, 4H), 7.51-7.64 (m, 4H), 7.60 (t, {\it J} = 11 Hz, 1H), 7.88 (dd, {\it J} = 22.5 Hz, {\it J} = 8.4 Hz, 4H), 8.18 (d, {\it J} = 8.7 Hz, 1H), 8.24 (d, {\it J} = 7.2 Hz, 1H), 8.60 (d, {\it J} = 8.7 Hz, 1H). \\

\noindent 
m/z of 876.2633 was obtained corresponding to a loss of N2, CH3, and 2H for the expected molecular formula C41H51N11O8S3. \\

%\clearpage

\subsection{{\it In sillico} selection of the clickable library}

Commercially available compounds from emolecules.com were downloaded and searched using ChemmineR for compounds containing a connection matrix (conMa) indicating the presence of a terminal acetylene functional group, this subset was further reduced using a drug-like filter. 

The emolecules database is pay per use, therefore for demonstration purposes we will detail equivalent operations on the freely accessible ZINC collection of 2.0 million compounds. The ZINC collection was batch downloaded using wget and uncompressed from sdf.gz and searched using the sdfstream function of ChemmineR. The sdfstream was performed to prepare a descriptor matrix of compounds in the ZINC collection, including the field indicating the presences of a terminal acetylene. Compounds containing terminal acetylenes were subset and denoted as ZINC80K in the figures. This subset can be filtered further using physiochemical features provided by ChemmineR, with extra features provided by Open Babel such as logP and Total Polar Surface Area (TPSA).

%\clearpage

\subsubsection{R code for selection of terminal acetylene subset from ZINC purchasible collection}

\begin{verbatim}
Bash code:
##########
wget -i list  
# where the list was provided for server path of 
# *.sdf.gz from zinc.docking.org/subsets/all-purchasable
# http://zinc.docking.org/db/bysubset/6/*.sdf.gz

R code:
##################
library(ChemmineR)
##################

#define function to collect descriptors

desc <- function(sdfset) {
        cbind(SDFID=sdfid(sdfset), 
              datablock2ma(datablocklist=datablock(sdfset)), 
              MW=MW(sdfset), 
              groups(sdfset), 
              AP=sdf2ap(sdfset, type="character"),
              rings(sdfset, type="count", upper=6, arom=TRUE)
        )
}

#execute sdfstream
sdfStream(input="ZINC_Purchasible.sdf", output="ZINC_Purchasible.xls", ...
append=FALSE, fct=desc, Nlines=1000)

#subset the sdf for compounds with terminal acetylenes
indexDF <- read.delim("ZINC_Purchasible.xls", row.names=1)
indexDFsub <- indexDF[indexDF$RCCH >= 1, ]

#subset source sdf with indexDFsub containing only the rows for compounds ...
with RCCH>=1

read.SDFindex(file="ZINC_Purchasible.sdf", index=indexDFsub, type="file", ...
outfile="ZINC_80K.sdf")

#This will write all molecules with RCCH>=1 to an SD file called "ZINC_80K.sdf".
#Help and comments provided by Thomas Girke
\end{verbatim}
\clearpage


\subsection{Subsetting ZINC purchasible}

The clickable collection was selected from ZINC80K by applying a combination of unique rule of five filters, shown using Open Babel, below. Alternatively, one may use physicochemical properties provided by ChemmineR shown in the next sections.

\begin{verbatim}
Bash code:
##########
babel ZINC_80K.sdf -osmi --filter "MW<450 logP > 5"
\end{verbatim}
%alternatively one may use babel to subset ZINC80K using a descriptor filter.

\subsection{Clustering and 3D-multidimensional scaling (3D-MDS) visualization of the clickable collection}
\begin{verbatim}
R code:
##################
library(ChemmineR)
##################
#Clickable.sdf provided as a record for the 4,002 compounds in the ...
#clickable collection of terminal acetylenes

#load sdfset and create apset for clustering
sdfset <- read.SDFset("Clickable.sdf")
apset <- sdf2ap(sdfset)

#cluster apset
clusters <- cmp.cluster(apset, cutoff = c(0.7))

#embed clusters in a 3-dimensional space based on apset
coord <- cluster.visualize(apset, clusters, size.cutoff=1, dimensions=3,...
quiet=TRUE)

#set ranges for colors
coord_alpha <- coord[1:2769,1:3]
coord_beta <- coord[2770:4002,1:3]

#############
library(rgl)
#############
rgl.open(); offset <- 50; par3d(windowRect=c(offset, offset, 640+offset, ...
640+offset))
rm(offset); rgl.clear(); rgl.viewpoint(theta=45, phi=30, fov=60, zoom=1)
spheres3d(coord_alpha[,1], coord_alpha[,2], coord_alpha[,3], radius=0.005, ...
color="black", alpha=1, shininess=20); aspect3d(1, 1, 1)
spheres3d(coord_beta[,1], coord_beta[,2], coord_beta[,3], radius=0.005, ...
color="red", alpha=1, shininess=20); aspect3d(1, 1, 1)
axes3d(col='black'); title3d("", "", "", "", "", col='black'); bg3d("white")  
############
rgl.snapshot("coord_alphaVsbeta.png")
############
rgl.close()
############
\end{verbatim}
\clearpage

\subsection{Generation of phyicochemical descriptors for SDF with Open Babel and ChemmineR}

\begin{verbatim}
Bash code:
##########
babel ZINC_80K.sdf -otxt --append TPSA > ZINC_80K.TPSA
babel ZINC_80K.sdf -otxt --append logP > ZINC_80K.logP

R code:
##################
library(ChemmineR)
##################

propma <- data.frame(MF=MF(sdfset, addH=FALSE), MW=MW(sdfset, addH=FALSE), ...
Ncharges=sapply(bonds(sdfset, type="charge"), length), atomcountMA(sdfset, ...
addH=FALSE), groups(sdfset, type="countMA"), rings(sdfset, upper=6, ...
type="count", arom=TRUE))
    
#add data from GetMoreProps.sh
More<-read.table(props[a])
names(More)<-c("TPSA","logP")
saved<-names(propma)
propma<-cbind(propma,More$TPSA, More$logP)
names(propma)<-c(saved,c("TPSA","logP"))
write.csv(propma, file=csv_name)

# propma is then sent for second treatment
######################
files <- list.files(recursive=TRUE, pattern = ".csv")
######################
par(mfrow=c(2,3)) 
######################
DoThis <- function(a){
######################
png_name<-paste(gsub(".csv","",files[a]),".png",sep="")
######################
png(file=png_name)
######################
MW <- read.csv(files[a])$MW
TPSA <- read.csv(files[a])$TPSA
logP <- read.csv(files[a])$logP
######################
library(scatterplot3d)
######################
scatterplot3d(MW,TPSA,logP, xlim=c(0,2000), ylim=c(0,800), ...
zlim=c(0,20), main=gsub(".csv","",files[a]))
######################
dev.off()
######################
}
######################
#a<-1:length(files)
a<-c(1,2,6,4,3,5)
lapply(a,DoThis)
######################
\end{verbatim}
%# To explore relationship between physicochemical descriptors load RShiny
\clearpage


\subsection{Physical assembly of the clickable library}

Click chemistry was chosen as the platform for our tagged library generation. Therefore, we assembled a diverse drug-like terminal acetylene library from vendors Asinex, Chembridge, Life Chemicals, Vitas-M, and Enamine. In total, 4,002 compounds were purchased for the clickable library. The unique molecular weight of each compound was used to add a corresponding volume of diluent, dimethylsulfoxide (DMSO), to each vile to a final concentration of 10 mM or 20 mM (for high molecular weight compounds). Dilution of the neat powders was automated using a Beckman -Coulter Biomex FXp liquid handler with a span-8 pod. Volume information was provided by using a comma separated value (csv) spreadsheet which determined the unique span-8 probe volume.

\subsection{{\it In sillico} synthesis of combinatorial libraries from azides and terminal acetylenes}

The program Reactor by ChemAxon was used to perform {\it in sillico} click reactions with terminal acetylenes and azides in .sdf or .mrv formats. A windows batch script written by Daniel Swank was made to perform the click reaction in a nested for loop, thereby creating a combinatorial library from all reactants. Two reaction files specifying the TBTA or Ruthenium catalyzed click reaction. \\

%consider simple case 4,000 x Block A

\clearpage
\begin{verbatim}
Batch script:
@echo off

echo ==========================
echo = Directory Reactor v0.1 =
echo ==========================


set reactcnt=0
set azidcnt=0
set alkcnt=0
set current=1
for %%A in (Reaction/*) do set /a reactcnt+=1
for %%A in (Azide/*) do set /a azidcnt+=1
for %%A in (Alkyne/*) do set /a alkcnt+=1
set /a count=%reactcnt% * %azidcnt% * %alkcnt%

FOR /f %%1 IN ('dir /b Reaction') DO (
FOR /f %%2 IN ('dir /b Azide') DO (
FOR /f %%3 IN ('dir /b Alkyne') DO (
set /a current+=1
echo Combining: %%~n2 + %%~n3 - %%~n1 [%current% of %count%]
call react -r Reaction/%%1 -m comb Azide/%%2 Alkyne/%%3 -f sdf -o ...
Output/%%~n2-%%~n3-%%~n1.sdf -i R1 -R Clk_ID -P PID -k
)
)
)
pause
\end{verbatim}

\clearpage

\subsection{Generation of fluorescent tagged libraries through click chemistry}

10 $\mu$l of 10 mM terminal acetylene was aliquotted to an empty 96-well polypropylene plate with 1.3 $\mu$l of 100 mM dye-azide. Thereafter, a 5 $\mu$l mixture derived from a 1:1:1 ratio of 10 mM TBTA-(CO2Na)3, 10 mM CuSO4, and 100 mM Na-ascorbate was added. Mixing was achieved by centrifugation followed by shaking overnight. Plates were allowed to incubate for three days prior to screening.


%%%%%%%%%%%%%%%%%%%%
\section{Results}
%%%%%%%%%%%%%%%%%%%%

\subsection{Synthesis of 2-azidoethan-1-amine}

\begin{figure}
\includegraphics[scale=0.30]{AEA_synth_cleaned}
\caption{Synthesis of 2-azidoethan-1-amine from 2-chloroethan-1-amine}
\label{fig:AEA}
\end{figure}

The synthesis of 2-azidoethan-1-amine was straightforward as described by literature, accomplished through azidolysis of 2-chloroethan-1-amine in water with some heat (Figure~\ref{fig:AEA}). 1HNMR data confirmed the presence of expected groups on the amine, with broad peaks at 1.43 ppm, and two expected triplets for alkyl CH2 groups. Note the CH2 group adjacent to the installed azide shifted downfield when compared to the precursor \cite{AEABenalil}.


Extraction of the product from the aqueous reaction mixture was optimized by titrating HCl produced during azidolysis of 2-chloroethan-1-amine with equimolar KOH, or until the pH was 8.5. The product was a volatile oil of high toxicity, and required extraction from the aqueous layer with ice cold diethyl ether; in addition, ice was used in the rotovap bath. A number of failed or low yield reactions resulted from the volatility of 2-azidoethan-1-amine. This product was used directly when possible, and decomposed readily. One may extend the shelf life of the oil by storing under a wrapped stopcock at 4 degrees in a vessel with a large head space. 

\begin{figure}
\includegraphics[scale=0.30]{DakinWest_synth_cleaned}
\caption{Conducting the Dakin-West reaction with glycine.}
\label{fig:DakinWest}
\end{figure}

%\clearpage

%\begin{landscape}

\begin{figure}
\centering
\includegraphics[scale=0.50]{DakinWestMechanism}
\caption{Mechanism of the Dakin-West reaction}
\label{fig:DakinWestMechanism}
\end{figure}

%\end{landscape}

\clearpage

\subsection{Synthesis of N-2-(oxopropyl)acetamide}

Our first building block N-2-(oxopropyl)acetamide (Figure~\ref{fig:DakinWest}) was synthesized in two steps from glycine starting material, as first described by Dakin and West \cite{DakinWest_Rxn}, and later by others \cite{Aminoacetone_Hepworth}. The reaction was first developed by Suvadeep Nath who conducted the Dakin-West reaction under reported conditions by boiling glycine and an excess of acetic anhydride and pyridine overnight under a nitrogen atmosphere. Under these conditions glycine is acetylated and cyclizes into a mesoionic heterocycle intermediate 1,3-oxazolium-5-olate or M{\"u}nchnone which reacts with excess acetic anhydride to give the desired product (Figure~\ref{fig:DakinWestMechanism}).

%add umlout

The resulting black crude product is isolated by removing the volatiles {\it in vacuo}, and the final product was isolated by hydrolyzing one acetyl group of the imide by refluxing in water (Figure~\ref{fig:DakinWestMechanism}). The extraction of the reaction mixture should be performed using a polar organic solvent such as dichloromethane, but the procedure is prolematic with high risk of emulsions and solubility problems exacerbated with residual pyridine. We later modified the reaction and boiled the Dakin-West product in methanol prior to workup (Figure~\ref{fig:DakinWest}).

The overall yield of the product is low from two to four percent. Considering the modest safety risks of exposure to refluxing acetyic anhydride and pyridine as well as the extraction procedure using large volumes of dichloromethane, the initial Dakin-West reaction is best done overnight as a very large volume and rarely as possible. If this approach is chosen, the crude product can be stockpiled and purified in batches as needed with no change in the product. An alternative to the workup has been devised that is more laborious, but produces a clear white oil that is difficult to obtain using silica purification alone.

We followed the doublet at 4.17 ppm in the 1HNMR to identify fractions containing the pure Dakin-West product. The impurities were removed through short path Kugelrohr distillation under reduced pressure, and remaining polar impurities were removed using silica gel and flash chromatography with a gradient elution in five steps from 98:2 to 90:10 using dichloromethane:isopropanol as eluent. The pure Dakin-West product shows a doublet at 4.17 with a coupling constant of 4.15 Hz for the CH2 group. The acetyl groups are seen as adjacent singlets at 2.05 and 2.22 ppm which represent the two amide rotamers. The amide shows a broad peak at 6.19 ppm. There is often a contaminant that has a doublet at 4.07 ppm that should not be mistaken for the product. Every effort to isolate a pure product including Reverse Phase High Pressure Liquid Chromatography (RP-HPLC) using a 50:50 acetonitrile:water isocratic method. 

\subsection{Synthesis of Block A}

\begin{figure}
\includegraphics[scale=0.30]{BlockA_synth_cleaned}
\caption{Synthesis of organic amine azide linker building block A}
\label{fig:BlockA}
\end{figure}

The product of the reductive amination of N-acetyl-N-(2-oxopropyl)acetamide with 2-azidoethan-1-amine resulted in the expected building block azide product N-N-2-[(2-azidoethyl)amino]propylacetamide (Figure~\ref{fig:BlockA}). It should be noted that this compound is sparingly solubule in chloroform and produces a poor 1HNMR spectrum at high concentrations. Therefore, ethyl acetate was chosen to workup the product. The stability of the azide under fluorescent lighting during handling is remarkable. 

1HNMR revealed that block A or N-N-2-[(2-azidoethyl)amino]propyl acetamide, hereafter Block A, has a methyl group with a chemical shift at 1.09 ppm. The amine is less de-shielded and is broad and upfield at 1.46 ppm. The acetyl group on the acetamide is unchanged from the initial Dakin-West product at 2.0 ppm. The two multiplets from 2.77-2.99 ppm are from similar CH2 groups next to the azide and acetamide, both with comparable electronegative neighbors.

%\clearpage

\subsection{Synthesis of building block B}

\begin{figure}
\includegraphics[scale=0.30]{BlockBPrecursor_cleaned}
\caption{Synthesis of building block B precursor}
\label{fig:BlockBPrecursor}
\end{figure}

\clearpage

\begin{figure}
\includegraphics[scale=0.30]{BlockB_synth_cleaned}
\caption{Synthesis of organic amine linker building block B}
\label{fig:BocBlockB}
\end{figure}

We will now discuss the synthesis of building block B from 1-aminopropan-2-ol conducted by Suvadeep Nath. The starting material was boc-protected in methanol to produce 1-[(tert-butoxy)amino]propan-2-ol, and the resulting alcohol was oxidized through a Swern oxidation to yield the product 1-[(tert-butoxy)amino]propan-2-one (Figure~\ref{fig:BlockBPrecursor}). Block B was synthesized by Suvadeep Nath similarly to Block A through a reductive amination with 2-azidoethan-1-amine (Figure~\ref{fig:BocBlockB}).

\subsection{Fluorophore conjugation to Block A and B}

\begin{figure}
\includegraphics[scale=0.30]{Panel8_cleaned}
\caption{Panel of fluorophores attached to block A. A-D. Coumarin dyes. E. Fluorescien isothiocyanate. F-H. Dyes available as sulfonyl chloride precursors.}
\label{fig:BlockADyes}
\end{figure}

Block A and Block B both possessed secondary amines and were amenable to fluorophore conjugation using standard methods. Suvadeep Nath prepared the coumarin and FITC block A derivatives (Figure 2.8A-E), whereas I synthesized dervatives from sulfonyl chloride precursors (Figure~\ref{fig:BlockADyes}F-H). The coumarin dyes were attached to building blocks A with ease using DCC coupling techniques via the addition of DMAP-HCl in dichloromethane at room temp. The dye, 7-(diethylamino)-2H-chromen-2-one (DEAC) was the first of the series (Figure~\ref{fig:BlockADyes}A). The second dye, 7-methoxy-2H-chromen-2-one contains a methoxy group on the coumarin fluorophore (Figure~\ref{fig:BlockADyes}B). The third fluorophore, 2H-benzo[h]chromen-2-one is an analog of the previous dye where a benzene ring provides the red shift for the fluorophore (Figure~\ref{fig:BlockADyes}C). The fourth, and last coumarin, 7-methyl-1H,2H,3H,4H-cyclopenta[c]chromen-4-one was the last coumarin synthesized by Suvadeep Nath (Figure~\ref{fig:BlockADyes}D).

\begin{figure}
\includegraphics[scale=0.30]{DansylBlockA_synth_cleaned}
\caption{Dansylation of block A}
\label{fig:DansylBlockA}
\end{figure}

Fluorescein isothiocyanate (FITC) conjugate of Block A was prepared by dissolving a building block and FITC in tetrahydrofuran (THC) since FITC is a known secondary amine nucleophile (Figure~\ref{fig:BlockADyes}E)(Suvadeep Nath). The final group of dyes were attached to block A from sulfonyl chloride precursors using DMAP, DIPEA in DCM (Figure~\ref{fig:BlockADyes}F-H). The first tye of this group was dibenzofuran (Figure~\ref{fig:BlockADyes}F). The second dye, 2-(thiophen-2-yl)pyridine from 5-(pyridin-2-yl)thiophene-2-sulfonyl chlorinde hereafter called pyridyl thiophene block A. The third dye, 5-chloro-2,1,3-benzoxadiazole-4-sulfonyl chloride, was also attached to block A. The structures of these three dyes was confirmed as click conjugates of clickable library acetylenes (Chapter 3: Chemical genomic screen).

The last dye attached to block A from a sulfonyl chloride precursor was N,N-dimethyl-5-sulfonylnaphthalene-1-amine (dansyl) chloride (Figure~\ref{fig:DansylBlockA}). This dye, when reacted with certain classes of the clickable library, lead to the most potent bioactive leads (Chapter 4: Phenotypes of bioactives). Therefore, the majority of this thesis pertains to the spectral characterization dansyl block A and block B dyes and select conjugates featured in later chapters (Chapter 4: Phenotypes of bioactives).

\begin{figure}
\includegraphics[scale=0.30]{DansylDiazirine_synth_cleaned}
\caption{Synthesis of dansyl diazirine building block B}
\label{fig:DansylDiazirine}
\end{figure}

The addition of the dansyl fluorophore to block A and B (Figure~\ref{fig:DansylDiazirine}) was accomplished through amine coupling methods in dry solvent and DMAP for a couple of hours. Contaminants were removed from the synthesis of dansyl block A using isocratic RP-HPLC conditions. 1HNMR confirmed dansyl azide amine block A had a sharp doublet at 0.96 ppm and several multiplets from 3-4 ppm  for the CH2 groups on block A. Aromatic protons for the dansyl dye were observed as expected.

\subsection{Synthesis of diazirine Block B}

The synthesis of diazirine functionalized dansyl amine azide building block B (Figure~\ref{fig:DansylDiazirine}) was accomplished by incubating the precursor in 10{\%} trifluoracetic acid in DCM for thirty minutes, to de-protect the amine. Volatiles were removed {\it in vacuo} and the free amine was created {\it in situ} \cite{kocienski2005protecting} through the addition of DIPEA to dissociate the amine-TFA salt. The resulting amine was reacted with commercially available N-hydroxysuccinimide ester alkyl diazirine or SDA-diazirine from Pierce.

The product, hereafter called dansyl daizrine amine azide (DDA) was purified from the reaction by using reverse phase chromatographic methods. The addition of the alkyl diazirine and linker to the building block caused an increase in retention time on the C8 column compared to the precursor (data not shown). HPLC purification was time intensive and low-yielding.  To remedy this I chromatographed DDA using a step inject procedure to minimize overall run time and maximize isolation of pure product (data not shown). Three closely spaced automated injections had sufficient resolution under our elution conditions, and was highly reproducible between runs. The starting material co-elutes prior to the desired product in reverse and normal phase methods, and may be an avenue for recovery if necessary.

The 1HNMR spectrum obtained was as expected, and the CH3 group on the diazirine ring is observed at the expected chemical shift with adjacent CH2 groups showing as a multiplet region upfield of 2.0 ppm. The synthesis of the alkyl and halogenated versions of the diazirine are discussed in the literature \cite{moss1995conversion,peptoids2010photoaffinity}\cite{mackinnon2007photo}. The biochemical appication of the diazirine ring has been discussed extensively \cite{work1979laboratory,bayley1983photogenerated} and will be discussed (Chatper 6: Target identification via Covalent capture). The stability of the compound is remarkable as the azide precursor and after the click reaction. The product remained intact and bioactive after several rounds of chromatography (Chapter 4: Phenotypes of bioactives). Nonetheless, precautions were taken to prevent exposure to sources of intense light and UV radiation.

\subsection{Synthesis of cell impermeable TBTA ligand}

\begin{figure}
\centering
\includegraphics[scale=0.27]{TBTA_synth_cleaned}
\caption{Synthesis of a cell impermiable TBTA-(CO2Na)3 catalyst for the click reaction.}
\label{fig:TBTA}
\end{figure}


A suitable TBTA ligand \cite{chan2004polytriazoles} was available from Sigam Aldrich to accelerate the copper catalyzed click reaction, but we chose to develop a novel cell impermiable version as a carboxylate salt. The TBTA variant enabled us to perform the biological screens of the click reactions directly. We included 10 micromolar EDTA into the growth media to chelate remaining copper. TBTA-(CO2Na)3 was synthesized by Suvadeep Nath using a novel route (Figure 2.11A-E) in five steps from 4-(chloromethyl)benzoic acid combined with 18-crown-6 and sodium azide (Figure~\ref{fig:TBTA}A). The product was extracted into diethyl ether with DMSO at room temperature, and reacted with acetyl chloride in methanol to give an esterified product (Figure~\ref{fig:TBTA}B). The product was treated with tri(propargyl)amine in standard click reactions. The ester was converted to a salt via base hydrolysis (Figure~\ref{fig:TBTA}C) using soidum hydroxide with heat in THF/methanol followed by gentle acid (Figure~\ref{fig:TBTA}D), and then an equimolar amount of NaOH was added to create the final charged carboxylate salt (Figure~\ref{fig:TBTA}E).

%show data

Biological testing confirmed TBTA-(CO2Na)3 was less bioactive than the ester and the carboxylic acid version. TBTA-(CO2Na)3 was tolerated up to 50 micromolar and higher (Chapter 3: Phenotypes of bioactives). The TBTA-(CO2Na)3 enabled us to perform bioassays directly on the click reactions with no purification (Chapter 3: Chemical genetic screen).

\subsection{{\it in sillico} selection of the clickable library}

\begin{figure}
\centering
\includegraphics[scale=0.55]{img15}
\caption{{\it in sillico} selection of the clickable library}
\label{fig:InSillicoSelection}
\end{figure}

The {\it in sillico} selection and automated assembly of the unique 4,002 compound library of drug-like terminal acetylenes was conducted using the R library(http://cran.r-project.org/) and ChemmineR \cite{cao2008chemminer} to select Rule Of Five (ROF) \cite{lipinski2000drug} compliant compounds with terminal acetylene functional groups from a collection of commercially available sources (emolecules.com) (Figure~\ref{fig:InSillicoSelection}). Since, e-molecules requires a subscription to download the entire compound database I will elucidate a similar method to search the freely available ZINC database.

ZINC is a host of approximately 2 million "cleaned" compounds in flexible chemical file formats (SDF, smiles, ihChI) along with availability and vendor info \cite{irwin2005zinc}. The library was downloaded as 49 gb of sdf.gz via wget and sent in chunks to the local 64-bit debian linux cluster (biocluster.ucr.edu) for parallel processing using R and the Bioconductor package ChemmineR \cite{cao2008chemminer}.

\subsection {Diversity of the clickable library}

%\clearpage
\begin{figure}
\centering
\includegraphics[scale=0.28]{Clickables_1vs2}
\caption{Three dimensional multidimensional scaling (3D-MDS) plot showing the chemical space of the clickable library. A. Black dots show the inital clickable collection of 2769 compounds. B. Red dots show the later addition to the library. C. The merge of the two collections in one MDS space.}
\label{fig:Clickables1vs2}
\end{figure}
%\clearpage

\begin{figure}
\includegraphics[scale=0.30]{PhysicoChemCompare}
\caption{Comparison of three physicochemical properties for molecular weight (MW), total polar surface area (TPSA), and solubility (logP) of several small molecule libraries available for screening at University of California Center for Plant Cell Biology (CEPCEB). A. 114 {\it Arabidopsis} bioactives in clickable collection (Chapter 3: Chemical genetic screen) occupy a small domain. B. The clickable collection of 4,002 terminal acetylenes. C. ZINC80K, a collection of 80,000 purchasable terminal acetylenes. D. Properties of expected products from a 38,000 member combinatorial library produced by performing high throughput reactions with the clickable library and a panel of fluorophores and one glucose azide (not shown). E. The collection of 72,000 diverse small molecules at the Center for Plant Cell Biology (CEPCEB) assembled from Chembridge Diverset, Chembridge Novacore, Sigma TimTec, and Life Chemicals. F. The spectrum library a 2,000 compound natural product derived from Microsource.}
\label{fig:PhysicochemicalCompare}
\end{figure}

The clickable library was assembled in two phases. In the first phase 2,769  compounds were arrayed in plate format, shown as red dots (Figure 2.13A) and the remaining 1,233 compounds were added to the collection later, shown as orange dots (Figure~\ref{fig:Clickables1vs2}B). Our acetylene search of ZINC determined the total purchasable chemical space with terminal acetylenes was greater than 80,000. We chose a subset of the whole for optimal "drug-likeness". The overall difference between the clickable collection, ZINC terminal acetylenes, and the resulting combinatorial libraries shown as merged, is apparent by comparison of the molecular weight, logP and total polar surface area physicochemical features (Figure~\ref{fig:PhysicochemicalCompare}).

\subsection {Physical assembly of the clickable library and combinatorial libraries}

\begin{figure}
\includegraphics[scale=0.50]{img18}
\caption{High throughput synthesis of fluorescent small molecule libraries form the clickable collection and azide building blocks.}
\label{fig:SynthFluorLibraries}
\end{figure}

The clickable library of 4,002 terminal acetylenes was purchased from commercial vendors as a neat powder, and diluted in DMSO to a 10 mM master stock using the Biomek FX (methods). Aliquots of the master plates were made for future chemical genetic screens and 1 microliter was transferred to polypropylene plates in duplicate with a fluorescently labeled dye-azide building block (Figure~\ref{fig:SynthFluorLibraries}A and B). One library was produced from each dye-azide linker, including one library from 1-beta-D-azido-glucopyranose. The click reaction was catalyzed through the addition of a 1:1:1 mixture of TBTA-(CO2Na)3 (10 mM) and CuSO4 (10 mM) and Na-ascorbate (100 mM) dissolved in water, hereafter called the click mix (Figure~\ref{fig:SynthFluorLibraries}C). Reactor software from ChemAxon (http://www.chemaxon.com) was used to generate a chemical library file with entries for each expected regiospecfific 1,2,3-substituted triazole and used for hit scoring, product prediction and record keeping.

Reactions were conducted for 3 days in the dark after centrifugation of the plates to collect all material in the center (Figure~\ref{fig:SynthFluorLibraries}D). The library was screened thereafter. It is well known compounds in diverse collections are not always soluble in DMSO \cite{lipinski2000drug} and may be less soluble in aqueous. Sonication helps but does not solve this problem. Quality control was conducted on random wells from different plates each day a new combinatorial library was made (Chapter 3: Chemical Genetic Screen). HPLC monitoring revealed the reaction took, on average 12-24 hours to reach full completion in our 96-well format. 

Precipitates appeared during the preparation combinatorial libaries from the clickables. This problem can be addressed systematically by removing the aqueous from the reaction {\it in vacuo} followed by re-suspsion in a minimum of DMSO.


\section{Conclusion}

We were successful to acquire 4,002 diverse clickable terminal acetylenes from vendors to build a large combinatorial library from click reactions with a fluorescent amine azide building block. The amine building block was synthesized from available precursors and we also synthesized a free amine building block as a precursor to a diazirine functionalized probe which will be highlighted later (Chapter 6: Target identification efforts by covalent modification). Furthermore, our combinatorial library was synthesized using TBTA-(CO2Na)3 to accelerate the screening process by testing the click reactions directly using {\it Arabidopsis thaliana} as a reporter (Chapter 3: Chemical Genetic Screen).


%CombinatorialChemistrySatisfied