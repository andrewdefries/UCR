\chapter{Target Identification efforts with a population Ler {\it Arabidopsis} M0}

\section{Abstract}

Previous chapters elucidate the discovery of several novel terminal acetylene and 1,2,3-triazole probes with an unknown mechanism of action. A resistance screen was conducted with two select bioactives to identify resistant candidates from a population of {\it Arabidopsis} M0. A quantitative approach was chosen to interrogate the resistant candidates, but strong resistant candidates were not identified for the bioactive and fluorescent 1,2,3-triazoles CLK024F02 dansyl block A and CLK042A09 dansyl block A. A discussion of future research to identify the target of these compounds is presented.

\section{Introduction} 

\subsection{EMS as a tool to identify small molecule targets in {\it Arabidopsis thaliana}}

Ethanemethylsulfonate (EMS) is a mutagen that has been utilized to introduce heritable mutations in {\it Arabidopsis} and isolate resistance alleles to novel probes \cite{EMS_Arabidopsis,park_2009}. EMS causes stable mutations by alkylating guanine of G-C pairs \cite{westhoff1998molecular}. The mutated base pair is replaced during replication or mismatch repair as an A-T pair, and the nucleotide change can alter the triplet codon to disrupt protein synthesis or to produce a non-functional protein. During the process of meiosis the mutant alleles are introduced to the germplasm from mutant sectors of the apical meristem that develop into the sillique structure \cite{westhoff1998molecular}. Screening the germplasm of EMS treated populations is an accepted method to find candidates with a genetic lesion involved in the mechanism of action of novel probes \cite{Zhao_Hyr1}. We will refer to this process in short as EMS screening.

\subsection{RFLP mapping of candidate mutants with a 30 cM primer walk}

\begin{figure}
\includegraphics[scale=0.50]{img74}
\caption{Figure adapted from Lukowitz et al. \cite{lukowitz2000positional}. A. Simple sequence length polymorphisms (SSLPs) exist between {\it Arabidopsis} ecotypes Col and Ler. B. Polymerase chain reaction (PCR) amplification of the polymorphic regions visualizes the nucleotide length difference for this allele in the homozygous Col and Ler backgrounds as well as the heterozygote. Std is a ladder with a range of 120-180 bp, Het indicates heterozygote.}
\label{fig:Lukowitz}
\end{figure}

Natural genetic variation between Columbia and Landsberg is a classical approach to determine the linkage of ecotype specific markers \cite{lukowitz2000positional}. The rough chromosomal location of a genetic lesion is found through mapping techniques and can refine down to a couple of open reading frames \cite{Art_of_Arabidopsis_Screens}. Mapping of an F2 population is accomplished through DNA extraction and PCR amplification of Simple Sequence Polymorphisms (SSLPs) (Figure ~\ref{fig:Lukowitz}A), where amplified DNA is subjected to 12{\%} agarose electrophoresis to determine the presence or absence of ecotype specific amplicons (Figure ~\ref{fig:Lukowitz}B) \cite{lukowitz2000positional}.

\section{Materials and Methods}


\subsection{Resistance screens with EMS library and bioactive probe}

EMS-mutagenized seed for the {\it Arabidopsis} Ler ecotype was soaked in mutagen and grown to maturity \cite{EMS_Arabidopsis}(Cutler lab). Seed was collected in segregated seed pools, to isolate siblings \cite{lightner199814seed}. Chlorotic sectors and lethality was observed in less than 1{\%} of seedlings screened, a typical observation of a successful mutagenesis procedure \cite{lightner199814seed}. Therefore, we used the seed pools for the identification of a resistant individual.

Assay plates were made from standard petri dishes filled with ms agar growth media, as previous, with the addition of compound to the final concentration of 50 $\mu$M for CLK024F02 dansyl block A plates and 25$\mu$M for CLK042A09 dansyl block A plates. Assay plates were chilled and grown in the dark as previous. Resistant candidates were identified with either long hypocotyls or roots or both and rescued on the final day using tweezers. Candidates that grew to maturity and set seed were re-tested under conditions identical to dose curves (Chapter 4: Phenotypes of Bioactives), and displayed as bagplot graphs. Discussion regarding bagplot graphs and significance values has been detailed in Chapter 3 at the opening of the Results section.

\section{Results}


\begin{figure}
\includegraphics[scale=0.50]{img75}
\caption{Schematic showing the steps involved to generate resistant mutant candidates from an EMS library of {\it Arabidospsis} seed. A. A resistant individual is identified from assay plates containing bioactive probe. B. The resistant candidate is rescued to normal growth media for a couple days and transferred to soil to produce more progeny for testing. C. {\it Arabidopsis} is allowed to grow to maturity and seed is collected when ready. D. Progeny from resistant individuals is tested in the original assay plate format with bioactive probe. E. The strongest candidates are selected, selfed or crossed to further molecular mapping strategies and assign dominance.}
\label{fig:EMS_MutantHowTo}
\end{figure}

\clearpage


\subsection{Baseline tests of CLK024F02 dansyl on Col and Ler}

\begin{figure}
\centering
\includegraphics[scale=0.65]{CLK024F02Dansyl_Baseline}
\caption{Baseline test of CLK024F02 dansyl on Col and Ler shown as blue and green bagplots. A. Both ecotypes shown on 1{\%} DMSO treatment. B. Treatment with 50 $\mu$M probe causes exagerrated changes in Ler and a drop in root values. C. Increased dose of 75 $\mu$M leads to a reduction in mean hypocotyl and root values (center of red star) for both Col (shown in blue) and Ler (shown in green).} 
\label{fig:CLK024F02Dansyl_Baseline}
\end{figure}

Initial tests with CLK024F02 dansyl on both ecotypes determined the expected range variation when dealing with two ecotypes as seen with the control values (Figure~\ref{fig:CLK024F02Dansyl_Baseline}A.). Nonetheless, the probe's effects are clear as the dose is increased to 50 and 75 $\mu$M causing an decrease in hypocotyl and root values from (8, 3.5) to (6,3) for Col and (7,4.5) to (4,3) for Ler. 

\subsection{The search for resistant candidates for CLK024F02 dansyl block A}

\begin{figure}
\centering
\includegraphics[scale=0.65]{Fig4_4}
\caption{Dose curve analysis of putative mutants identified by EMS screening. A-C. DMSO control. D-F. Doses are indicated 50 $\mu$M (light blue) and 75 $\mu$M (red) for the lines shown.} 
\label{fig:Fig4_4}
\end{figure}

The resistant candidate CLK024F02D-SC-43-1 had a typical growth on DMSO with inhibitory effects observed at 50 and 75 $\mu$M (Figure~\ref{fig:Fig4_4}A and D). Notice the characteristics of the bag and loop hull are retained at the lower dose of 50 $\mu$M (Figure~\ref{fig:Fig4_4}D). CLK024F02D-SC-43-2 has a wide spread of high values at 75 $\mu$M, but the mean has dropped from (8,5) to (6,4) (Figure~\ref{fig:Fig4_4}B). CLK024F02D-SC-43-3, the last line in this panel has a marked reduction in mean values at 50 and 75 $\mu$M (Figure~\ref{fig:Fig4_4}F).

\begin{figure}
\centering
\includegraphics[scale=0.65]{Fig4_5}
\caption{Dose curve analysis of putative mutants identified by EMS screening. A-C. DMSO control. D-F. Doses are indicated 50 $\mu$M (light blue) and 75 $\mu$M (red) for the lines shown.} 
\label{fig:Fig4_5}
\end{figure}

\begin{figure}
\centering
\includegraphics[scale=0.65]{Fig4_6}
\caption{Dose curve analysis of putative mutants identified by EMS screening. A-C. DMSO control. D-F. Doses are indicated 50 $\mu$M (light blue) and 75 $\mu$M (red) for the lines shown.} 
\label{fig:Fig4_6}
\end{figure}

The mean values of CLK024F02D-SC-44-1 appear unchanged at 50 and 75 $\mu$M with some outliers showing values of (2,2) (Figure~\ref{fig:Fig4_5}A and D). Similarly the mean of CLK024F02D-SC-44-2 is not changed at 75 $\mu$M, and out performs control values at 50 $\mu$M (Figure~\ref{fig:Fig4_5}B and E). The last line in this panel, CLK024F02D-SC-45-2 was scarce and only data for 50 $\mu$M was derived to demonstrate top performers are slightly inhibited at this dose (Figure~\ref{fig:Fig4_5}C and F). The performance of CLK024F02D-SC-45-2 can be compared to CLK024F02D-SC-46-1 and CLK024F02D-SC-46-2 at a similar dose. Both CLK024F02D-SC-46-1 and CLK024F02D-SC-46-2 have values larger than for respective DMSO control (Figure~\ref{fig:Fig4_6}A-B, D-E). CLK024F02D-SC-49-1 on the other hand has a bivariate mean at (6,4) that drops slightly to (5,3.5) (Figure~\ref{fig:Fig4_6}C and F).

\subsection{The search for resistant candidates for CLK042A09 dansyl block A}

\begin{figure}
\centering
\includegraphics[scale=0.65]{CLK042A09Dansyl_Baseline}
\caption{Baseline test of CLK042A09 dansyl on Col and Ler shown as blue and green bagplots. A. Both ecotypes shown on 1{\%} DMSO treatment. B. Treatment with 25 $\mu$M probe causes exaggerated changes in Ler and a drop in root values. C. Increased dose of 50 $\mu$M leads to a reduction in mean hypocotyl and root values (center of red star) for both Col (blue) and Ler (green).} 
\label{fig:CLK042A09Dansyl_Baseline}
\end{figure}

\begin{figure}
\centering
\includegraphics[scale=0.65]{Fig4_8}
\caption{Dose curve analysis of putative mutants identified by EMS screening. A-C. DMSO control. D-F. Doses are indicated 25 $\mu$M (blue) and 50$\mu$M (red) for the lines shown.} 
\label{fig:Fig4_8}
\end{figure}

CLK042A09 dansyl block A is potent to restrict the hypocotyl and root values significantly at 25 $\mu$M with subtle changes up to 50 $\mu$M (Figure~\ref{fig:CLK042A09Dansyl_Baseline}A and B.). CLK042A09D-SC-53-2 was a promising candidate with a subtle change in bivariate mean from the control values (6,3) to (6,2.5) when dosed at 25 and 50 $\mu$M (Figure~\ref{fig:Fig4_8}A and D). The line CLK042A09D-SC-53-2 had a number of candidates with long roots and average sized hypocotyls (Figure~\ref{fig:Fig4_8}D). The next candidate CLK042A09D-SC-53-3 performed well at 25 and 50 $\mu$M with a majority of the sample having mean values above control for the line (Figure~\ref{fig:Fig4_8}B and E). The last in this panel CLK042A09D-SC-53-4 has a reduction in root growth at 50 $\mu$M with a significant proportion with short roots and hypocotyls (Figure~\ref{fig:Fig4_8}C and F).

\begin{figure}
\centering
\includegraphics[scale=0.65]{Fig4_9}
\caption{Dose curve analysis of putative mutants identified by EMS screening. A-C. DMSO control. D-F. Doses are indicated 25 $\mu$M (blue) and 50 $\mu$M (red) for the lines shown.} 
\label{fig:Fig4_9}
\end{figure}

CLK042A09D-SC-54-1 had a clear reduction in hypocotyl and root values at 25 and 50 $\mu$M despite having a good performance on DMSO (Figure~\ref{fig:Fig4_9}A and D). The line CLK042A09D-SC-54-2 had unchanged mean values at 25 and 50 $\mu$M for the hypocotyl and a reduction in root size (Figure~\ref{fig:Fig4_9}B and E). The last line CLK042A09D-SC-55 performed poorly on DMSO and on 50 $\mu$M (Figure~\ref{fig:Fig4_9}C and F). The values for 25 $\mu$M were well below the means for Col and Ler (Figure~\ref{fig:Fig4_9}F) compared to (Figure~\ref{fig:Fig4_8}B). The results (Figure~\ref{fig:Fig4_4}-Figure~\ref{fig:Fig4_9}) were derived from the set of Ler-EMS mutant candidates rescued during the first resistance screen (Figure~\ref{fig:EMS_MutantHowTo}C).


%\clearpage
\begin{figure}
\centering
\includegraphics[scale=0.75]{PutantAcetylene_01}
\caption{Dose curve analysis of putative mutants identified by EMS screening. A-C. DMSO control. D-F. Doses are indicated 12 $\mu$M (yellow), 25 $\mu$M (brown) and 50 $\mu$M (red) for the lines shown.} 
\label{fig:Fig4_10}
\end{figure}

\begin{figure}
\centering
\includegraphics[scale=0.75]{PutantAcetylene_02}
\caption{Dose curve analysis of putative mutants identified by EMS screening. A-C. DMSO control. D-F. Doses are indicated 12 $\mu$M (yellow), 25 $\mu$M (brown) and 50 $\mu$M (red) for the lines shown.} 
\label{fig:Fig4_11}
\end{figure}

\begin{figure}
\centering
\includegraphics[scale=0.75]{PutantAcetylene_03}
\caption{Dose curve analysis of putative mutants identified by EMS screening. A-C. DMSO control. D-F. Doses are indicated 12 $\mu$M (yellow), 25 $\mu$M (brown) and 50 $\mu$M (red) for the lines shown.} 
\label{fig:Fig4_12}
\end{figure}

\begin{figure}
\centering
\includegraphics[scale=0.75]{PutantAcetylene_04}
\caption{Dose curve analysis of putative mutants identified by EMS screening. A-C. DMSO control. D-F. Doses are indicated 12 $\mu$M (yellow), 25 $\mu$M (brown) and 50 $\mu$M (red) for the lines shown.} 
\label{fig:Fig4_13}
\end{figure}

\begin{figure}
\centering
\includegraphics[scale=0.75]{PutantAcetylene_05}
\caption{Dose curve analysis of putative mutants identified by EMS screening. A-C. DMSO control. D-F. Doses are indicated 12 $\mu$M (yellow), 25 $\mu$M (brown) and 50 $\mu$M (red) for the lines shown.} 
\label{fig:Fig4_14}
\end{figure}

\begin{figure}
\centering
\includegraphics[scale=0.75]{PutantAcetylene_06}
\caption{Dose curve analysis of putative mutants identified by EMS screening. A-C. DMSO control. D-F. Doses are indicated 12 $\mu$M (yellow), 25 $\mu$M (brown) and 50 $\mu$M (red) for the lines shown.} 
\label{fig:Fig4_15}
\end{figure}

\clearpage

\subsection{Secondary search for resistant candidates to CLK024F02 dansyl block A and CLK042A09 dansyl block A}


A second resistance screen was conducted to identify more resistant individuals to CLK024F02 dansyl and CLK042A09 dansyl (Figure~\ref{fig:EMS_MutantHowTo}B). A number of candidates were found, rescued and grown to maturity to acquire seed for analysis (Figure~\ref{fig:EMS_MutantHowTo}B and C). Due to limited availability of the probes CLK024F02 dansyl and CLK042A09 dansyl the progeny of the rescued Ler-EMS seedlings were first interrogated using the respective terminal acetylene fragments for any signs of cross-resistance to the acetylene fragment (Figure~\ref{fig:Fig4_10})-(Figure~\ref{fig:Fig4_15}). 

The line CLK042A09D-SC-49-3-1 has mean values centered around (150, 100) for all doses of CLK042A09 examined (Figure~\ref{fig:Fig4_10}A and D). Line CLK042A09D-SC-56-2-2 has very tall hypocotyls for all doses examined with a reduction of root values showing at 30 $\mu$M (Figure~\ref{fig:Fig4_10}B and E). The performance of CLK042A09D-SC-50-5 on 30 $\mu$M CLK042A09 shows a large variance not observed for the DMSO control (Figure~\ref{fig:Fig4_10}C and F). The outliers for this sample at 30 $\mu$M almost segregate the bagplot into two regions (Figure~\ref{fig:Fig4_10}F). In one extreme, the bagplot for CLK042A09D-SC-50-5 has a number of individuals with shorter roots and long hypocotyls (Figure~\ref{fig:Fig4_10}F, top half of baghull) and the other has long roots with short hypocotyls (Figure~\ref{fig:Fig4_10}F, bottom half of baghull). 

All doses of CLK042A09 caused a reduction of the hypocotyl and root bagplot domains of CLK042A09D-SC-60-3 with no reduction in the bivariate mean (Figure~\ref{fig:Fig4_11}A and D). The line CLK042A09D-SC-49-5 has long roots on 15 $\mu$M wihich shorten when the dose is doubled (Figure~\ref{fig:Fig4_11}E). The shape and span of the baplot for CLK042A09-SC-57-5 at 30 $\mu$M (Figure~\ref{fig:Fig4_11}F) is similar to CLK042A09D-SC-49-5 (Figure~\ref{fig:Fig4_11}E) at a lower dose. Several outliers can be seen at the forefront of the bagplot for CLK042A09D-SC-57-5 at a high dose (Figure~\ref{fig:Fig4_11}F). 

In the next panel the line CLK042A09D-SC-57-4 appears to have long roots at 5 $\mu$M of CLK042A09 which shortens to a similar domain for 15 and 30 $\mu$M (Figure~\ref{fig:Fig4_12}A and D). Performance of the line CLK042A09-SC-57-2-2 at 30 $\mu$M (Figure~\ref{fig:Fig4_12}B and E) looks promising, but is suspicious since lengths at 15 $\mu$M were so short (Figure~\ref{fig:Fig4_12}E). The next line, CLK042A09D-SC-49-4 appears similar to previous with a slightly better performance at 15 $\mu$M (Figure~\ref{fig:Fig4_12}F). The subtle differences between these lines can be elucidated with future analysis. In comparison to other panels (Figure~\ref{fig:Fig4_13})-(Figure~\ref{fig:Fig4_14}A-F) the bagplots may indicate resistance trends worth investigating. 

\subsection{Clear phenotypes that are difficult to interpret}

In the first scenario CLK042A09D-SC-46-5 has a gradual compression of both hypocotyl and root domains which are nested in each other (Figure~\ref{fig:Fig4_15}D). As we increase the dose {\it Arabidopsis} has a gradual reduction in overall length. In the second example, the mean is slightly higher than control for all doses and the low dose has outliers with taller hypocotyls (Figure~\ref{fig:Fig4_15}B and E). An increase of the dose from 5 to 15 $\mu$M causes the expected reduction in values which do not change further when doubling the dose (Figure~\ref{fig:Fig4_15}E). The third and last scenario is a peculiar one where the middle dose outperforms the control values and the high dose has compressed the baghull of control values (Figure~\ref{fig:Fig4_15}F). The mechanics of the resistance performances in these rescued lines of {\it Arabidopsis} may be resolved with similar resistance analysis using closely spaced doses. 

\section{Summary and Discussion}

%\subsection{Identification of a resistant candidate to Clade M germination inhibitor compound CLK003D03}

%Resistant candidates to the clade M germination inhibitor CLK003D03 was identified in the M0 resistance screen. The resistant mutant was isolated from a Ler M0 population (Figure ~\ref{fig:EMS_MutantHowTo}A-B) and crossed to Col. The progeny were scored for the germination phenotype when treated with compound (Figure ~\ref{fig:EMS_MutantHowTo}C-D). Phenotype scores were used to determine the dominance of the mutation, and linkage was established using SSLPs \cite{EMS_Arabidopsis}(Jayati Mandal, Cutler lab). Molecular markers in combination with whole genome sequencing were utilized to build a chromosomal filter to identify the causative SNP by base calling (data not shown, in Press). DNA for the Illumina read was extracted from the original F2 CLK003D03 resistant individual used for mapping and chromosome filter. Adult floral tissue was macerated and the DNA was sonicated to 450 bps and ligated to adapters for attachment to the flow cell. The allele responsible was found to be a mutated Fumarylacetoacetate hydrolase (FAAH). It is hypothesized that the FAAH conjugates the precursor, CLK003D03, into a bioactive acid form (in press). All other candidate bioactives and mutants were not investigated to this depth as the phenotypes were not as simple as germination inhibition, a binary phenotype.

\subsection{Sequencing options to identify the causative resistance allele to bioactives}

A resistant candidate for the 1,2,3-triazole probes was not identified during screening. A number of options are available to perform candidate gene sequencing once a resistant individual is identified. Illumina array based sequencing \cite{ossowski2010rate} was the method chosen to identify the SNP for CLK003D03.  The traditional Sanger di-deoxy sequencing technique \cite{bentley2008accurate} and other high throughput sequencing platforms such as 454 are available to identify genetic lesions caused by EMS-mutagenesis. The approach chosen should consider the number of candidates to test, budget, and extend of secondary genetic lesions.

\subsection{Considerations before Illumina sequencing}

SHOREMAP \cite{ossowski2010rate} has been used to find genetic lesions in {\it Arabidopsis}. Austin et al. \cite{austin2011next} demonstrated that we can identify causative SNPs by Illumina sequencing of allelic pooled F2 populations. Rough mapping was used to determine linkage of the resistance allele to a rough chromosomal location. In addition, tests were made to determine a sufficient level of sufficient genetic coverage when sequencing pools of samples using Illumina to accomplish successful SNP identification \cite{austin2011next}. In addition, the F2 line may be selfed several generations prior to mapping to remove secondary lesions. This consideration may become the norm since Austin et al. and others have reported genetic heterogeneity at the SNP level in homozygous inbred populations and even between lab strains of the Columbia-0 ecotype. 

\subsection{Quantitative phenotype metrics offer a solution to evaluate individuals and a population {\it en masse}}

Strict quantitative phenotype evaluation of individuals may clarify the decision making process when planning costly whole genome sequencing of resistant mutant, mutant candidates or pools. Especially when the bioactive causes a subtle phenotype or the probe is very rare. Furthermore, sound quantitative data may be used for statistical analysis of mutant pools to discover causative SNPs. 

%\section{Discussion}

\section{Conclusion}

My goal was to identify resistant {\it Arabidopsis} candidates that could be used to perform map based target identification strategies for the rare probes CLK024F02 dansyl block A and CLK042A09 dansyl block A. Several resistance screens were performed and the results were visualized using a bivariate bagplot. A number of candidates display trends of resistance for both or either of hypocotyl and root values, but these were not explored further. 

The phenotypes of CLK024F02 dansyl and CLK042A09 dansyl were previously unknown. Since these probes can be modified to diazirine containing photo-affinity ligands (Chapter 3 and Chapter 4) they satisfied our primary objective in a probe, despite the modest phenotype. Therefore, we quantitatively demonstrated that individuals can be found with dose specific trends of resistance. These trends may not be apparent or discernible by eye. Therefore, by systematizing the collection and display of these values in bagplots we can describe the range and behavior of novel candidates from EMS pools. A threshold for resistance will be a necessary parameter for further resistance screening for either of hypocotyl or root values, and tests should be done to confirm if dose specific resistance can be linked to a resistance allele. Seed for select individuals will be submitted to the Arabidopsis Biological Resource Center (ABRC)\cite{lamesch2012arabidopsis} to support external validation.  

