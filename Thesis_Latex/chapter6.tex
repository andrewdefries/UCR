\chapter{Target identification efforts via covalent capture}

\section{Abstract}

The diazirine ring was explored for use as a photo-affinity module in our fluorescent and bioactive probes. 1,2,3-Triazole conjugates were prepared from dansyl diazirine block B in the TBTA catalyzed click reaction (Chapter 2: Organic Synthesis). The bioactive candidate CLK042A09 dansyl diazirine was the focus of our investigation. A preliminary experiment was conducted to label {\it Arabidopsis thaliana} protein extracted from etiolated seedlings.  Non-specific labeling of a small molecular weight protein was observed for CLK042A09 dansyl diazirine and the inactive analog CLK039G03 dansyl diazirine. Experiments to label generic protein such as Bovine Serum Albumin (BSA) was conducted to learn more regarding the stoichiometry of ligand-protein binding. These experiments were inconclusive. To simplify our experiment we chose to label single amino acids with the probe CLK042A09 dansyl diazirine. In sum, our results suggested activation of the diazirine probe by ultraviolet photolysis at 365 nm was insufficient. Enhancement of photolysis conditions using a Mercury arc lamp was confirmed by 1HNMR. Probe was rare and this was the extent of examination permissible.

\section{Introduction}

Our strategy was to identify bioactive probes that can be modified into diazirine functionalized versions for target identification strategies without altering potency. The probe CLK042A09 dansyl diazirine was more potent than CLK024F02 dansyl diazirine and selected for use (synthesized in Chapter 2: Organic Synthesis and described quantitatively in Chapter 4: Phenotypes of Bioactives). Given the novelty of the probe there was no {\it a priori} knowledge of the target. Nonetheless, a number of situations can be hypothesized regarding the potential biological target(s) of CLK042A09 dansyl diazirine along with the most logical means to detect these proteins (Table \ref{table:HypotheticalSituations}). Due to space limitations the most desirable category of probe, an IC50 of high (nM) with high target abundance has a high chance of success for affinity capture and coomasie stain. A number of experiments were performed to identify the target protein of CLK042A09 dansyl diazirine {\it in vivo}. Factors such as target abundance and potency were of primary concern, but out of our hands. 

\begin{table}\centering
    \begin{tabular}{|l|l|l|l|l|l|}
    \hline
    IC50 & High (nM)  & Low ($\mu$M) & Low ($\mu$M) & ~               & ~                 \\ \hline
    Target abundance        & Low        & High         & Low          & ~               & ~                 \\ \hline
    ~                       & Likelihood & of           & success      & Range           & Probe 
\\ \hline
    Affinity capture        & Good       & Poor         & Poor         & ~               & Biotin tag        \\ \hline
    Coomassie stain         & Maybe      & Poor         & Unlikely     & 30-100 ng       & None              \\ \hline
    Fluorescence            & Good       & Poor         & Unlikely     & ?               & Fluoresce    \\ \hline
    Silver stain            & Good       & Poor         & Unlikely     & 5-10 ng         & None              \\ \hline
    Chemiluminescence       & Good       & Good         & Maybe        & 0.25-1.0 ng     & Biotin tag        \\ \hline
    \end{tabular}
    \caption {Hypothetical situations regarding detection of a unknown protein target for a novel bioactive small molecule.}
    \label{table:HypotheticalSituations}
\end{table}
\clearpage

\subsection{Diazirine is a small photo-activatible probe}

\begin{figure}\centering
\includegraphics[scale=0.50]{img76}
\caption{Synthesis of diazirine from precursors.}
\label{fig:DiazirineSynthLit}
\end{figure}

Diazirine rings may be prepared in three steps for a myriad of functional groups (Figure ~\ref{fig:DiazirineSynthLit}A-C) \cite{work1979laboratory,moss1995conversion}. Diazirine is formed from starting material in the presence of ammonia and chloroamine which is then oxidized to form an energetic diazirine ring \cite{moss1995conversion, peptoids2010photoaffinity} \cite{tanaka2008photoactivatable}. The product may be photolysed using UV light of 365 nm to liberate nitrogen to produce a carbene {\it in situ} (Figure ~\ref{fig:DiazirineSynthLit}D.)\cite{klan2009photochemistry}. The short lived carbene is used to label protein(s). Non ideal conditions may cause the generation of an undesired diazo (Figure ~\ref{fig:DiazirineSynthLit}D) that produces re-arrangement products \cite{bergmann1994hexestrol, work1979laboratory}.

\subsection{Diazirine for covalent capture of small molecule targets}

Covalent capture of protein targets {\it in vivo} or {\it in vitro} is a complementary route to target identification of novel small molecules. Precending chapters lay the groundwork for use of diazirine functionalized photo affinity ligands (PALs) of CLK024F02 dansyl and CLK042A09 dansyl.

\subsection{Promising diazirine probes for use in covalent capture}

Dose curve analysis determined that both CLK024F02 dansyl diazirine and CLK042A09 dansyl diazirine both retained activity profiles similar to the block A versions. These results were consistent with the expectation that subtle changes on the acetamide would not disturb bioactivity.The analog to CLK042A09, CLK039G03 dansyl block A, was demonstrated as inactive (Figure ~\ref{fig:CLK042A09DoseCurvePretty}). Therefore, the compounds CLK042A09 dansyl diazirine and CLK039G03 may be used in combination during biochemical experiments as positive and negative controls.

%\section{Materials and Methods}
%\subsection{PBS}
%\subsection{BSA}
%\subsection{Ampule test}

\section{Results}

\subsection{Experiments to covalently label phosphate buffered saline (PBS) extracted {\it Arabidopsis} protein}

Encouraged by the bioactivity of the diazirine functionalized probes we performed biochemical labeling experiments on {\it Arabidopsis} proteins extracted with phosphate buffered saline in Laemmli buffer. The PALs were added to a final concentration of 100 $\mu$M and added to the protein sample. Protein samples were split into two pools, one for UV exposure, and the other as a control. Samples were irradiated with UV light for 20 minutes on ice, and run on a 12{\%} sodium dodecyl sulfate polyacrylamide gel electrophoresis (12{\%}-SDS-PAGE).

Experiments resulted in the labeling of a small abundant 35-55 kDa sized protein for all chemical classes. Early attempts with the diazirine probe heed caution for the interpretation of photolysis experiments where small abundant proteins are sequestered non-specifically \cite{work1979laboratory}. Removal of small abundant proteins may shed light on the observed non-specific labelling. Subsequent experiments were more conservative with the goal to improve photolysis conditions and improve product formation.

\subsection{Attempts to covalently label BSA}

Investigation of the efficiency of the probe CLK042A09 dansyl diazirine to covalently label samples of Bovine Serum Albumin (BSA). Photolysis conditions and electrophoresis conditions were identical to the previous experiment with 20 minute UV exposure at 3 cm distance with a 20 watt source. These conditions were sufficient for activation of our PAL \cite{klan2009photochemistry}. Despite previous efforts labeling protein in PBS with Laemmli buffer, we were not able to label BSA. The PAL dye was seed to run through the gel ahead of the smallest 66.5 KDa band. It was suspected that the exclusion of Laemmli buffer lead to inefficient labeling, but an insufficient amount of probe remained to continue the investigation.

\subsection{Attempts to covalently label select amino acids}

\begin{figure}
\centering
\includegraphics[scale=0.50]{img77}
\caption{Scheme showing expected result for small scale labeling experiment with CLK042A09 dansyl diazirine.}
\label{fig:DiazirineLabelAA}
\end{figure}

The remainder of CLK042A09 dansyl diazirine was used in three separate small scale reactions to label glycine, tryptophan, and phenylalanine. The samples were irradiated with UV light for 20 minutes at a distance of less than 1 cm, and the product was subjected to ESI LC/MS (Figure ~\ref{fig:DiazirineLabelAA}. Mass analysis confirmed the the presence of starting material, but this is inconclusive. No specific new product was detected.

\subsection{Enhancement of photolysis conditions using a Mercury arc lamp and optical filter}

Rather than exploring a longer duration of photolysis we sought to use a more energetic UV light source. An inverted Leica microscope and an optical filter was used to selectively transmit light of 365 nm from an Olympus Mercury arc lamp. The sample was dissolved in D6-DMSO in an NMR tube and 1HNMR was recorded before and after exposure to UV light for 20 minutes. 1HNMR analysis determined full diazirine photolysis. 

\section{Summary}

The type of alkyl diazirine tested, NHS-diazirine, is one of a family of related photo-affinty ligands. Our results can not be generalized to all the variants, especially since the literature has documented subtle enhancements of labeling through replacement of the terminal methyl group on alkyl diazirine to trifluoromethyl \cite{huss2002concanamycin,zhang1994location} \cite{blencowe2005development,brunner19803}. One report of a {\it p}-[(3-trifluoromethyl)diaizirine-3-yl]benzoic acid was shown to have a half-life of approximately two minutes after exposure to UV light \cite{hatanaka1994novel}. In addition, a myriad of covalent labels, available as NHS-esters, may have been explored using our building block B such as the epoxide group (K-trap) and benzophenone \cite{krauth2009heterobifunctional}. 

\section{Discussion}

\section{Covalent probes and amino acid reactivity}

\begin{figure}\centering
\includegraphics[scale=0.57]{AA_modify}
\caption{Twenty common amino acids are shown with potential sites for reactivity circled. In this original figure different colors represent potential insertion events. It is unknown how the sites in each amino acid would react to a covalent probe as a monomer or as peptide. This figure is an attempt to exhaustively enumerate groups in each amino acid as candidates for alteration by a photoaffinity probe.}
\label{fig:AA_modify}
\end{figure}

We generated a number of unique probes and attempted to use them in the most crude way possible. A systematic study of the reactivity of each amino acid, as monomers, and thereafter as peptides would be the best starting point to characterize the CLK024F02 and CLK042A09 dansyl diazirine covalent probes. A number of sites exist in amino acids that may react with a covalent probe. A systematic study could determine the philicity of a probe to a particular bond or group (Figure ~\ref{fig:AA_modify}). These studies could provide deep insight into this problem.
%philicity ref

\subsection{Biotin affinity probes}

The precursor to the diazirine functionalized 1,2,3-triazoles can potentially react with iodo-acetyl-PEG-biotin thus providing a affinity group rather than a covalent group. The biotinylated molecule may be used in biochemical pull down efforts for target enrichment, and the diazirine probe may be used afterward to label the enriched pools. Furthermore, a biotin probe may be used in combination with phage display to enrich antigens \cite{smith1985filamentous}, and has been successful in {\it Arabidopsis} to find FK506 binding proteins using a biotinylated FK506 probe \cite{piggott2009rapid}.


\section{Conclusion}

Labelling studies may be more successful using a Mercury arc lamp with optical filter. The use of NMR to confirm photolysis prior to mass analysis may reveal more information without complicating the investigation. A systematic set of experiments must be conducted to produce a model of how {\it in situ} created carbenes from photolyzed PALs can be efficiently utilized. Tests with different solvents, additives, and conditions must be conducted \cite{brunner19803}. The use of microprobe NMR may aid in the minimization of probe used per experiment.

\begin{figure}
\centering
\includegraphics[scale=0.50]{img78}
\caption{Illustration demonstrating the application of photo affinity ligands (PALs) for unbiased genomic protein studies using protein chips from the Arabidopsis Biological Resource Center (ABRC). A-C. Preparation of the chip. D-E. Proposed steps to label proteins using our PALs.}
\label{fig:DiazirineonABRCchips}
\end{figure}

More work is necessary to characterize the bioactive PALs CLK024F02 dansyl diazirine and CLK042A09 diazirine for successful biochemical target capture. A number of issues remain as primary concerns for the investigator. One, target abundance is of significance for our probe with a modest IC50 (Figure 4.13 and 4.17). In these instances detection could be enhanced through chemiluminescent detection biotin using avidin-linked horse radish peroxidase (Figure 6.3).

Cheap protein chips are available from the ABRC with standardized coverage genomic translation products (Figure ~\ref{fig:DiazirineonABRCchips}) \cite{zhu2003protein}. On these protein arrays cross-linked cDNA {\it in vitro} translated using cell free translation systems leaving protein adhered to the chip. There is potential to use this platform for the identification of covalently labelled proteins, and discount a number of concerns that arise from labeling of unpurified protein from a whole organism.

Even though this study was inconclusive regarding identifying the target, an example has been presented in {\it Arabidopsis} {\it in vivo} where activity is preserved when an acetyl group of the acetamide is replaced by that diazirine acylating agent (Figure ~\ref{fig:CLK024F02DD}) and (Figure ~\ref{fig:CLK042A09DD}). This thesis is a demonstration that we can develop photo affinity ligands from fragments of existing probes with modest changes to potency, and a little ingenuity. 